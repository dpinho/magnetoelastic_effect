\chapter{Acoplamento Magneto-e\'alstico}

\section{Introdu\c{c}\~ao}

Como vimos na subse\c{c}\~ao \ref{sec.fund_eletr}, quando uma estrutura condutiva se movimenta num campo magn\'etico, uma corrente el\'etrica e um campo magn\'etico vari\'avel s\~ao gerados nessa estrutura. Segundo \cite{Mikhailenko_1997}, a passagem de uma onda s\'ismica pela subsuperf\'icie terrestre gera o movimento do material que comp\~oe essa subsuperf\'icie. Considerando que esse material \'e cont\'inuo e el\'astico, cont\'em uma certa distribui\c{c}\~ao de cargas el\'etricas e que o planeta Terra possui um campo geomagn\'etico natural, temos que o movimento relativo entre o material e o campo geomagn\'etico vai gerar varia\c{c}\~oes geomagn\'eticas locais associadas \`as ondas s\'ismicas que provocaram o movimento do material. Mais ainda, segundo \cite{Anisimov_1985} e \cite{Sadovsky_1980}, a onda eletromagn\'etica induzida \'e ''congelada`` \`a onda s\'ismica e se propaga n\~ao com a velocidade da luz, mas com a velocidade da onda $P$ ou da onda $S$, dependendo do tipo de onda s\'ismica, e podemos registrar e estudar essa varia\c{c}\~ao geomagn\'etica.
Um corpo nessas condi\c{c}\~oes \'e chamado de s\'olidos eletromagn\'etico-el\'asticos, essas varia\c{c}\~oes no campo geomagn\'etico s\~ao chamadas de \textit{ondas sismomagn\'eticas} e esse efeito recebeu o nome de \textit{efeito sismomagn\'etico} ou \textit{efeito magnetoel\'astico}.

Segundo \cite{erigen_1963}, \'e poss\'ivel investigar algumas intera\c{c}\~oes din\^amicas que podem ocorrer entre campos eletromagn\'eticos e campos el\'asticos em s\'olidos homog\^eneos e isotr\'opicos. Assim, esses autores desenvolveram um modelo matem\'atico de combina\c{c}\~ao entre a teoria de elasticidade infinitesimal e a teoria eletromagn\'etica linearizada, o qual ser\'a apresentado a seguir. A intera\c{c}\~ao se deve principalmente \`a for\c{c}a de corpo de Lorentz, \`a modifica\c{c}\~ao das equa\c{c}\~oes constitutivas e das condi\c{c}\~oes de contorno provocada pela velocidade do material, e \`as for\c{c}as superficiais introduzidas pelos campos.

\section{Equa\c{c}\~oes Constitutivas do Meio}
Seguindo a nota\c{c}\~ao apresentada em \cite{erigen_1963}, temos que os vetores eletromagn\'eticos que representam o meio de propaga\c{c}\~ao das ondas propriamnte dito s\~ao representados por $\mathbf{E}^0$, $\mathbf{D}^0$, $\mathbf{B}^0$, $\mathbf{H}^0$ e
$\mathbf{J}^0$. Essas mesmas quantidades, quando se referirem \`as medidas observadas em laborat\'orio s\~ao denotadas apenas retirando-se o sobre-escrito $0$. Transferindo essa nota\c{c}\~ao para as equa\c{c}\~oes \ref{eq.D_funcao_E} e \ref{eq.B_funcao_H} temos
\begin{equation}\label{eq.constitutivas_parciais}
\mathbf{D}^0=\epsilon\,\mathbf{E}^0\quad\text{e}\quad\mathbf{B}^0=\mu\,\mathbf{H}^0.
\end{equation}

Em muitos materiais, a densidade de corrente el\'etrica \'e linearmente dependente de um campo el\'etrico externo, e tal rela\c{c}\~ao, conhecida como a \textit{lei de Ohm}, pode ser escrita usando a nota\c{c}\~ao apresentada acima como
\begin{equation}\label{eq.lei_ohm}
\mathbf{J}^0=\sigma\,\mathbf{E}^0,
\end{equation}
onde $\sigma$ \'e a \textit{condutividade} do meio. As equa\c{c}\~oes \ref{eq.constitutivas_parciais} juntamente com a equa\c{c}\~ao \ref{eq.lei_ohm} s\~ao denominadas \textit{equa\c{c}\~oes eletromagn\'eticas contitutivas do meio}, quando o mesmo \'e isotr\'opico, homog\^eneo e se encontra em repouso. As mesmas equa\c{c}\~oes se mant\^em quando o meio se move por ocasi\~ao da passagem de uma onda s\'ismica. 


\section{Modelo de Dunkin e Erigen}






\section{Conclusões}