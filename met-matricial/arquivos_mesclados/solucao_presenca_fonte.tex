\chapter{Solu\c{c}\~ao na Presen\c{c}a de Fonte}\label{sec.presenca_fonte}
Em porspec\c{c}\~ao de petr\'oleo s\~ao utilizadas alguns tipos de fontes de ondas, atrav\'es das quais se faz um mapeamento das caracter\'isticas das camadas de subsuperf\'icie. Esses tipos de fontes podem ser uma queda de peso, um caminh\~ao \textit{vibroseis}, explosivos e canh\~ao de ar, esta \'ultima fonte utilizada em prospec\c{c}\~ao mar\'itima. Sendo assim, vamos desenvolver uma solu\c{c}\~ao para o nosso problema considerando agora a presen\c{c}a de uma fonte.

Considere ainda a equa\c{c}\~ao \ref{eq.matricial} com o sobrescrito $m$ omitido. Uma fonte $\mathbf{S}$ localizada numa profundidade $z_s$ pode ser representada na forma
\begin{equation}\label{eq.fonte_geral}
\mathbf{S}=\mathbf{S}_0\delta(z-z_s)+\mathbf{S}_1\delta^\prime(z-z_s),
\end{equation}
onde $\mathbf{S}_0$ e $\mathbf{S}_1$ n\~ao dependem da profundidade e $\delta$ \'e a fun\c{c}ao \textit{Delta de Dirac}. Fontes que s\~ao distribu\'idas ao longo da profundidade podem, geralmente, ser sintetizadas por superposi\c{c}\~ao de fontes do tipo $\mathbf{S}_0$ e $\mathbf{S}_1$. 

Uma solu\c{c}\~ao por ser escrita como a combina\c{c}\~ao de uma solu\c{c}\~ao inicial sofrendo a a\c{c}\~ao de alguma fonte, ou seja,
\begin{equation}\label{eq.solucao_inicial_fonte}
\mathbf{\Phi}=\mathbf{\Phi}_0+\mathbf{S}_1\delta(z-z_s).
\end{equation} 
Substituindo a equa\c{c}\~ao \ref{eq.solucao_inicial_fonte} e a equa\c{c}\~ao \ref{eq.fonte_geral} na equa\c{c}\~ao \ref{eq.matricial}, temos
\begin{equation}\label{eq.matricial_fonte}
\frac{d\,\mathbf{\Phi}_0}{d\,z}=-i\,\omega\,M\,\mathbf{\Phi}_0+\left[\mathbf{S}_0-i\,\omega\,M\,\mathbf{S}_1\right]\,\delta(z-z_s),
\end{equation}
e por simplicidade, vamos escrever
\begin{equation}\label{eq.fonte_simplificada}
i\,\omega\,M\,\mathbf{S}_1-\mathbf{S}_0=
\begin{pmatrix}
\mathbf{S}_A\\
\mathbf{S}_B
\end{pmatrix}.
\end{equation}
Considerando a exist\^encia de uma interface imagin\'aria na profundidade $z_s$ da fonte, podemos determinar as condi\c{c}\~oes de salto no local da fonte da mesma forma que estudamos as condi\c{c}\~oes de salto nas interfaces que separam as camadas. Assim, integrando a equa\c{c}\~ao \ref{eq.matricial_fonte} no intervalo que come\c{c}a imediatamente acima da interface imagin\'aria da fonte $z_s^-$, e termina imediatamente abaixo da interface imagin\'aria da fonte em $z_s^+$, e substituindo a equa\c{c}\~ao \ref{eq.fonte_simplificada}, temos como solu\c{c}\~ao
\begin{equation*}
\mathbf{\Phi}_0(z_s^-)=\mathbf{\Phi}_0(z_s^+)+
\begin{pmatrix}
\mathbf{S}_A\\
\mathbf{S}_B
\end{pmatrix}.
\end{equation*}
Substituindo a equa\c{c}\~ao \ref{eq.solucao_inicial_fonte} e considerando as caracter\'isticas da fun\c{c}\~ao Delta de Dirac, temos a seguinte condi\c{c}\~ao de salto na profundidade da fonte
\begin{equation}\label{eq.salto_zs}
\mathbf{\Phi}(z_s^-)=\mathbf{\Phi}(z_s^+)+
\begin{pmatrix}
\mathbf{S}_A\\
\mathbf{S}_B
\end{pmatrix}.
\end{equation}

Vamos agora inserir uma interface imagin\'aria imediatamente abaixo da fonte, em $z=z_s^+$ e utilizar os m\'etodos do cap\'itulo \ref{sec.ausencia_fonte} para computar a matriz de reflex\~ao $\Gamma_s\equiv\Gamma(z_s^+)$ a partir do topo desta camada. J\'a que a interface em $z_s^+$ eh fict\'icia, as propriedades do meio s\~ao iguais acima e abaixo dessa interface, assim temos que $L_2^+=L_2^-$ e $L_1^+=L_1^-$. Substituindo essas identidades nas equa\c{c}\~oes \ref{eq.j_a} e \ref{eq.j_b}, temos
\begin{align}\label{eq.j_a_ficticia}
J_A&=\frac{1}{2}\left[(L_2)^\top L_1+(L_1)^\top L_2\right]\\\label{eq.j_b_ficticia}
J_B&=\frac{1}{2}\left[(L_2)^\top L_1-(L_1)^\top L_2\right].
\end{align}
Pela diagonaliza\c{c}\~ao de Ursin apresentada nas subse\c{c}\~oes \ref{sec.diagonalizacao_1}-\ref{sec.diagonalizacao_4}, conclu\'imos que 
\begin{align*}
L_1^\top&=L_2^{-1}\\
L_2^\top&=L_1^{-1},
\end{align*}
e substituindo as equa\c{c}\~oes acima nas equa\c{c}\~oes \ref{eq.j_a_ficticia} e \ref{eq.j_b_ficticia}, obtemos que
\begin{align*}
J_A&=I\\
J_B&=0.
\end{align*}
Desta forma, a onda ascendente $\mathbf{U}(z_s^+)$ e a onda descendente $\mathbf{D}(z_s^+)$ a partir da interface em $z_s^+$ podem ser introduzidas na equa\c{c}\~ao \ref{eq.definicao_psi} para obtermos
\begin{equation*}
\mathbf{\Psi}(z_s^+)=
\begin{pmatrix}
\mathbf{U}(z_s^+)\\
\mathbf{D}(z_s^+)
\end{pmatrix}.
\end{equation*}
Substituindo a equa\c{c}\~ao \ref{eq.reflexao_m+1} na equa\c{c}\~ao acima, temos
\begin{equation}\label{eq.Psi_descendente}
\mathbf{\Psi}(z_s^+)=
\begin{pmatrix}
\Gamma_s\mathbf{D}(z_s^+)\\
\mathbf{D}(z_s^+)
\end{pmatrix},
\end{equation}
pois as ondas est\~ao numa mesma camada e da\'i usamos que
\begin{align*}
\mathbf{U}^-(z_s^+)&=\mathbf{U}(z_s^+)\\
\mathbf{D}^-(z_s^+)&=\mathbf{D}(z_s^+).
\end{align*}
Multiplicando a equa\c{c}\~ao \ref{eq.salto_zs} por $L^{-1}$ e substituindo a equa\c{c}\~ao \ref{eq.Phi}, obtemos
\begin{equation}\label{eq.Psi_salto_zs}
\mathbf{\Psi}(z_s^-)=\mathbf{\Psi}(z_s^+)+L^{-1}
\begin{pmatrix}
\mathbf{S}_A\\
\mathbf{S}_B
\end{pmatrix}.
\end{equation}
Generalizando os desenvolvimentos apresentados nas subse\c{c}\~oes \ref{sec.diagonalizacao_1}-\ref{sec.diagonalizacao_4}, podemos escrever 
\begin{equation*}
L^{-1}=\frac{1}{\sqrt{2}}
\begin{pmatrix}
L_2^\top&L_1^\top\\
L_2^\top&-L_1^\top
\end{pmatrix},
\end{equation*}
e substituindo esta express\~ao para $L^{-1}$ juntamente com a equa\c{c}\~ao \ref{eq.Psi_descendente} na equa\c{c}\~ao \ref{eq.Psi_salto_zs}, temos
\begin{equation}\label{eq.solucao_phi_zs-}
\mathbf{\Psi}(z_s^-)=
\begin{pmatrix}
\Gamma_s\mathbf{D}(z_s^+)\\
\mathbf{D}(z_s^+)
\end{pmatrix}
+
\frac{1}{\sqrt{2}}\,
\begin{pmatrix}
L_2^\top\mathbf{S}_A+L_1^\top\mathbf{S}_B\\
L_2^\top\mathbf{S}_A-L_1^\top\mathbf{S}_B
\end{pmatrix}.
\end{equation}
Admitindo que a fonte esteja no interior da primeira camada, ou seja, $0<z_s<z_1$, a solu\c{c}\~ao dada pela equa\c{c}\~ao \ref{eq.solucao_phi_zs-} \'e propagada para cima a partir de $z_s^-$ usando a equa\c{c}\~ao \ref{eq.solucao_psi}, e o salto atrav\'es das interfaces entre camadas \'e dado pela equa\c{c}\~ao \ref{eq.psi_matriz_salto} at\'e que a onda atinja a interface terra/ar em $z=0^+$. Assim,
\begin{equation*}
\mathbf{\Psi}(0^+)=e^{-i\,\omega\,\tilde{\mathbf{\Lambda}}\,(0^+-z_s^-)}\,\mathbf{\Psi}(z_s^-),
\end{equation*}
e podemos usar as $n$ condi\c{c}\~oes de fronteira em $z=0$ para determinarmos as $n$ inc\'ognitas de $\mathbf{D}_s$. Os demais termos da solu\c{c}\~ao s\~ao conhecidos. A diferen\c{c}a $z_s^--0^+$ corresponde \`a profundidade da fonte, assim a solu\c{c}\~ao anterior pode ser reescrita como
\begin{align*}
\mathbf{\Psi}(0^+)&=e^{-i\,\omega\,\tilde{\mathbf{\Lambda}}\,(-z_s)}\,\mathbf{\Psi}(z_s^-)\\\\
\mathbf{\Psi}(0^+)&=
\begin{pmatrix}
e^{i\,\omega\,\mathbf{\Lambda}\,z_s}&\mathbf{0}\\
\mathbf{0}&e^{-i\,\omega\,\mathbf{\Lambda}\,z_s}
\end{pmatrix}
\mathbf{\Psi}(z_s^-).
\end{align*}
Substituindo a equa\c{c}\~ao \ref{eq.solucao_phi_zs-} na equa\c{c}\~ao acima, temos
\begin{equation}\label{eq.Psi_zero+}
\mathbf{\Psi}(0^+)=
\begin{pmatrix}
e^{i\,\omega\,\mathbf{\Lambda}\,z_s}\,\Gamma_s\mathbf{D}(z_s^+)\\
e^{-i\,\omega\,\mathbf{\Lambda}\,z_s}\,\mathbf{D}(z_s^+)
\end{pmatrix}
+
\frac{1}{\sqrt{2}}\,
\begin{pmatrix}
e^{i\,\omega\,\mathbf{\Lambda}\,z_s}\,(L_2^\top\mathbf{S}_A+L_1^\top\mathbf{S}_B)\\
e^{-i\,\omega\,\mathbf{\Lambda}\,z_s}\,(L_2^\top\mathbf{S}_A-L_1^\top\mathbf{S}_B)
\end{pmatrix}.
\end{equation}

TALVEZ SEJA INTERESSANTE COLOCAR A PARTE SEGUINTE NO CAPITULO ONDE SE DESENVOLVE AS EQUACOES MAIS PARTICULARES.

Para an\'alises futuras ser\'a interessante escrever a solu\c{c}\~ao na superf\'icie como
\begin{equation}\label{eq.Phi_G_AB}
\mathbf{\Phi}(0^+)=
\begin{pmatrix}
G_A\mathbf{\Phi}_g\\
G_B\mathbf{\Phi}_g
\end{pmatrix},
\end{equation}
onde as matrizes $G_A$ $G_B$ s\~ao de dimens\~ao $n\times n$ e $\mathbf{\Phi}_g$ \'e um vetor de dimens\~ao $n$ formado por inc\'ognitas em $z=0$.
Considerando o sistema \ref{eq.matricial_1}, temos
\begin{equation*}
\mathbf{\Phi}^{(1)}=
\begin{pmatrix}
\tilde{E}_1\\
\tilde{H}_2
\end{pmatrix}
\end{equation*}
e sua condi\c{c}\~ao de fronteira, quando $z=0$ \'e dada pela equa\c{c}\~ao \ref{eq.cond_fron_1}, 
\begin{equation*}
\tilde{H}_2=-\frac{\epsilon_0}{q_0}\tilde{E}_1.
\end{equation*}
Substituindo esta condi\c{c}\~ao em $\mathbf{\Phi}^{(1)}$, temos
\begin{equation*}
\mathbf{\Phi}^{(1)}=
\begin{pmatrix}
\tilde{E}_1\\\\
-\frac{\epsilon_0}{q_0}\tilde{E}_1
\end{pmatrix}_{z=0^+}.
\end{equation*}
Colocando a equa\c{c}\~ao acima no formato da equa\c{c}\~ao \ref{eq.Phi_G_AB}, temos que
\begin{equation*}
\mathbf{\Phi}_g^{(1)}=\tilde{E}_1,
\end{equation*}
\begin{equation*}
G_A^{(1)}=1\qquad\text{e}\qquad G_B^{(1)}=-\frac{\epsilon_0}{q_0}.
\end{equation*}
De maneira an\'aloga, vamos escrever as solu\c{c}\~oes dos tr\^es sistemas restantes na forma dada pela equa\c{c}\~ao \ref{eq.Phi_G_AB}. Substituindo a condi\c{c}\~ao de contorno \ref{eq.cond_fron_2} no sistema \ref{eq.matricial_2}, $\mathbf{\Phi}^{(2)}$ \'e dado por
\begin{equation*}
\mathbf{\Phi}^{(2)}=
\begin{pmatrix}
\tilde{E}_2\\\\
-\frac{q_0}{\mu_0}\tilde{E}_2
\end{pmatrix}_{z=0^+}
\end{equation*}
e temos
\begin{equation*}
\mathbf{\Phi}_g^{(2)}=\tilde{E}_2,
\end{equation*}
\begin{equation*}
G_A^{(2)}=1\qquad\text{e}\qquad G_B^{(2)}=-\frac{q_0}{\mu_0}.
\end{equation*}
Substituindo a condi\c{c}\~ao de contorno \ref{eq.cond_fron_3} no sistema \ref{eq.matricial_3}, $\mathbf{\Phi}^{(3)}$ \'e dado por
\begin{equation*}
\mathbf{\Phi}^{(3)}=
\begin{pmatrix}
\dot{\tilde{u}}_3\\
0\\
0\\
\dot{\tilde{u}}_1
\end{pmatrix}_{z=0^+}
\end{equation*}
e temos,
\begin{equation*}
\mathbf{\Phi}_g^{(3)}=
\begin{pmatrix}
\dot{\tilde{u}}_3\\
\dot{\tilde{u}}_1
\end{pmatrix},
\end{equation*}
\begin{equation*}
G_A^{(3)}=
\begin{pmatrix}
1&0\\
0&0
\end{pmatrix}
\qquad\text{e}\qquad G_B^{(3)}=
\begin{pmatrix}
0&0\\
0&1
\end{pmatrix}.
\end{equation*}

Substituindo a condi\c{c}\~ao de contorno \ref{eq.cond_fron_4} no sistema \ref{eq.matricial_4}, $\mathbf{\Phi}^{(4)}$ \'e dado por
\begin{equation*}
\mathbf{\Phi}^{(4)}=
\begin{pmatrix}
\dot{\tilde{u}}_2\\
0
\end{pmatrix}_{z=0^+}
\end{equation*}
e temos,
\begin{equation*}
\mathbf{\Phi}_g^{(4)}=\dot{\tilde{u}}_2,
\end{equation*}
\begin{equation*}
G_A^{(4)}=1\qquad\text{e}\qquad G_B^{(4)}=0.
\end{equation*}

De acordo com a equa\c{c}\~ao \ref{eq.Phi}, podemos escrever
\begin{equation*}
\mathbf{\Phi}(0^+)=L\,\mathbf{\Psi}(0^+),
\end{equation*}
e substituindo as equa\c{c}\~oes \ref{eq.Psi_zero+}, \ref{eq.Phi_G_AB} e \ref{eq.matriz_L} na equa\c{c}\~ao acima, temos
\begin{align}\label{eq.GA_Phi}
G_A\mathbf{\Phi}_g&=\frac{1}{\sqrt{2}}\,(L_1e^{i\,\omega\,\Lambda\,z_s}\Gamma_s\mathbf{D}_s+L_1e^{-i\,\omega\,\Lambda\,z_s}\mathbf{D}_s)\\\nonumber
&+\frac{1}{2}\,\left[L_1e^{i\,\omega\,\Lambda\,z_s}(L_2^\top\mathbf{S}_A+L_1^\top\mathbf{S}_B)+L_1e^{-i\,\omega\,\Lambda\,z_s}(L_2^\top\mathbf{S}_A-L_1^\top\mathbf{S}_B)\right]
\end{align}
e
\begin{align}\label{eq.GB_Phi}
G_B\mathbf{\Phi}_g&=\frac{1}{\sqrt{2}}\,(L_2e^{i\,\omega\,\Lambda\,z_s}\Gamma_s\mathbf{D}_s-L_2e^{-i\,\omega\,\Lambda\,z_s}\mathbf{D}_s)\\\nonumber
&+\frac{1}{2}\,\left[L_2e^{i\,\omega\,\Lambda\,z_s}(L_2^\top\mathbf{S}_A+L_1^\top\mathbf{S}_B)-L_2e^{-i\,\omega\,\Lambda\,z_s}(L_2^\top\mathbf{S}_A-L_1^\top\mathbf{S}_B)\right].
\end{align}
Pela diagonaliza\c{c}\~ao de Ursin podemos verificar que $L_1^\top=L_2^{-1}$, e multiplicando pela esquerda a equa\c{c}\~ao \ref{eq.GA_Phi} por $L_2^\top$ e multiplicando pela esquerda a equa\c{c}\~ao \ref{eq.GB_Phi} por $L_1^\top$, temos 
\begin{align}\label{eq.LGA_Phi}
L_2^{\top}G_A\mathbf{\Phi}_g&=\frac{1}{\sqrt{2}}\,(e^{i\,\omega\,\Lambda\,z_s}\Gamma_s\mathbf{D}_s+e^{-i\,\omega\,\Lambda\,z_s}\mathbf{D}_s)\\\nonumber
&+\frac{1}{2}\,\left[e^{i\,\omega\,\Lambda\,z_s}(L_2^\top\mathbf{S}_A+L_1^\top\mathbf{S}_B)+e^{-i\,\omega\,\Lambda\,z_s}(L_2^\top\mathbf{S}_A-L_1^\top\mathbf{S}_B)\right]
\end{align}
e
\begin{align}\label{eq.LGB_Phi}
L_1^\top G_B\mathbf{\Phi}_g&=\frac{1}{\sqrt{2}}\,(e^{i\,\omega\,\Lambda\,z_s}\Gamma_s\mathbf{D}_s-e^{-i\,\omega\,\Lambda\,z_s}\mathbf{D}_s)\\\nonumber
&+\frac{1}{2}\,\left[e^{i\,\omega\,\Lambda\,z_s}(L_2^\top\mathbf{S}_A+L_1^\top\mathbf{S}_B)-e^{-i\,\omega\,\Lambda\,z_s}(L_2^\top\mathbf{S}_A-L_1^\top\mathbf{S}_B)\right].
\end{align}
Subtraindo a equa\c{c}\~ao \ref{eq.LGB_Phi} da equa\c{c}\~ao \ref{eq.LGA_Phi}, temos
\begin{equation}\label{eq.pre_phi_g}
(L_2^{\top}G_A-L_1^\top G_B)\mathbf{\Phi}_g=\frac{2}{\sqrt{2}}\,e^{-i\,\omega\,\Lambda\,z_s}\mathbf{D}_s+e^{-i\,\omega\,\Lambda\,z_s}(L_2^\top\mathbf{S}_A-L_1^\top\mathbf{S}_B).
\end{equation}
Isolando $\mathbf{D}_s$ obtemos
\begin{equation*}
\mathbf{D}_s=\frac{1}{\sqrt{2}}\,e^{i\,\omega\,\Lambda\,z_s}(L_2^{\top}G_A-L_1^\top G_B)\,\mathbf{\Phi}_g-\frac{1}{\sqrt{2}}(L_2^\top\mathbf{S}_A-L_1^\top\mathbf{S}_B).
\end{equation*}
Ao multiplicar pela esquerda a equa\c{c}\~ao \ref{eq.pre_phi_g} por $e^{i\,\omega\,\Lambda\,z_s}\,\Gamma_se^{i\,\omega\,\Lambda\,z_s}$, temos
\begin{equation}\label{eq.pre_phi_g2}
e^{i\,\omega\,\Lambda\,z_s}\,\Gamma_se^{i\,\omega\,\Lambda\,z_s}(L_2^{\top}G_A-L_1^\top G_B)\mathbf{\Phi}_g=\frac{2}{\sqrt{2}}\,e^{i\,\omega\,\Lambda\,z_s}\,\Gamma_s\mathbf{D}_s+e^{i\,\omega\,\Lambda\,z_s}\,\Gamma_s(L_2^\top\mathbf{S}_A-L_1^\top\mathbf{S}_B),
\end{equation}
e somando as equa\c{c}\~oes \ref{eq.LGA_Phi} e \ref{eq.LGB_Phi}, temos
\begin{equation}\label{eq.pre_phi_g+}
(L_2^{\top}G_A+L_1^\top G_B)\mathbf{\Phi}_g=\frac{2}{\sqrt{2}}\,e^{i\,\omega\,\Lambda\,z_s}\Gamma_s\mathbf{D}_s+e^{i\,\omega\,\Lambda\,z_s}(L_2^\top\mathbf{S}_A+L_1^\top\mathbf{S}_B).
\end{equation}
Subtraindo a equa\c{c}\~ao \ref{eq.pre_phi_g+} da equa\c{c}\~ao \ref{eq.pre_phi_g2}, obtemos uma rela\c{c}\~ao para $\mathbf{\Phi}_g$, dada por 
\begin{align*}
\mathbf{\Phi}_g&=\left[e^{i\,\omega\,\Lambda\,z_s}\,\Gamma_se^{i\,\omega\,\Lambda\,z_s}(L_2^{\top}G_A-L_1^\top G_B)-(L_2^{\top}G_A+L_1^\top G_B)\right]^{-1}\\
&\,\,\cdot\,\,e^{i\,\omega\,\Lambda\,z_s}\left[\Gamma_s (L_2^\top\mathbf{S}_A-L_1^\top\mathbf{S}_B)-(L_2^\top\mathbf{S}_A+L_1^\top\mathbf{S}_B)\right].
\end{align*}
Em particular, quando a fonte est\'a imediatamente abaixo da superf\'icie livre, $z_s\approx0$, temos
\begin{equation*}
\mathbf{\Phi}_g=\left[(\Gamma_s-I)L_2^{\top}G_A-(\Gamma_s+I)L_1^\top G_B\right]^{-1}\left[(\Gamma_s-I)L_2^\top\mathbf{S}_A-(\Gamma_s+I)L_1^\top\mathbf{S}_B\right].
\end{equation*} 
Ap\'os obtermos $\mathbf{\Phi}_g$, \'e poss\'ivel determinarmos todas as condi\c{c}\~oes iniciais em $z=0$, $\mathbf{D}_s$ e $\mathbf{U}_s=\Gamma\,\mathbf{D}_s$. Em seguida, podemos obter a solu\c{c}\~ao imediatamente abaixo da fonte conforme a rela\c{c}\~ao \ref{eq.Psi_descendente}. Teoricamente, a partir de agora, a solu\c{c}\~ao pode ser computada em qualquer outra profundidade utilizando a propaga\c{c}\~ao atrav\'es das camadas de acordo com a rela\c{c}\~ao \ref{eq.solucao_psi} e o salto atrav\'es das camadas usando a rela\c{c}\~ao \ref{eq.psi_matriz_salto}. No entando, a propaga\c{c}\~ao de uma onda ascendente cont\'inua no sentido descendente \'e numericamente inst\'avel usando a equa\c{c}\~ao \ref{eq.solucao_psi}, pois as exponenciais complexas crescem ao inv\'es de diminuirem com a dist\^ancia. Assim, devemos obter $\mathbf{U}$ a partir de $\mathbf{D}$ usando $\Gamma_m$, ou fazer uso das matrizes de transmiss\~ao $T_m$.


