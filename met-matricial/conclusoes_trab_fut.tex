\chapter{Conclus\~oes e Trabalhos Futuros}
Neste trabalho apresentamos um tratamento matem\'atico das EDP's do efeito magneto-el\'astico encontrado em \cite{pinho_2018} , no sentido de propiciar a contru\c{c}\~ao de um algoritmo num\'erico est\'avel que possa descrever a propaga\c{c}\~ao acoplada de ondas el\'asticas e eletromagn\'eticas. Nesse tratamento foi fundamental a aplica\c{c}\~ao de conhecimentos da F\'isica-Matem\'atica, Geof\'isica e, em particular, um metodo matricial que facilita a an\'alise de propaga\c{c}\~ao de ondas em meios estratificados.

Vimos na subse\c{c}\~ao \ref{sec.matricial_poroelast} a possibilidade de an\'alise de dispers\~ao de atenua\c{c}\~ao de ondas para casos diversos, onde tal an\'alise auxilia na verifica\c{c}\~ao e constru\c{c}\~ao de um c\'odigo computacional efetivo para descrever a propaga\c{c}\~ao dessas ondas. Numa oportuinidade futura, queremos aplicar a an\'alise de atenua\c{c}\~ao e dispers\~ao nesse sistema de EDP's do efeito magneto-el\'astico com a finalidade de ajudar a estudar o comportamento da propaga\c{c}\~ao.

Numa determinada abordagem, a an\'alise de casos mais simples auxilia no estudo de casos mais sofisticados. Por tanto, no intuito ainda de otimizar o estudo da propaga\c{c}\~ao das ondas, faremos o tratamento matem\'atico das EDP's de magneto-elasticidade para o caso unidimensional, considerando a propaga\c{c}\~ao em fun\c{c}\~ao do tempo e em fun\c{c}\~ao da profundidade. Neste caso podemos utilizar o m\'etodo matricial e a an\'alise de atenua\c{c}\~ao e dispers\~ao das ondas, e economizamos a utiliza\c{c}\~ao de transformadas e mudan\c{c}a de eixos coordenados.

O formato final das EDO's dado no cap\'itulo \ref{sec.trans_edp_2_edo} apresentou algumas vari\'aveis incluidas como fonte, diferentemente do que \'e preconizado por Ursin, onde todas a vari\'aveis devem estar inseridas no vetor $\mathbf{\Phi}$. Assim, analisaremos a possibilidade da aplica\c{c}\~ao de fun\c{c}\~oes de Green juntamente com o m\'etodo matricial para contornar esse problema. \'E poss\'ivel que essa abordagem traga desafios computacionais consider\'aveis e da\'i estudaremos tamb\'em outras alternativas. Uma delas \'e considerar o efeito magento-el\'astico para o caso totalmente acoplado e verificar se o novo formato das equa\c{c}\~oes permite a exclus\~ao de vari\'avies dadas como fonte. Outra possibilidade \'e escrever as equa\c{c}\~oes em coordenadas cil\'indricas, considerar as propriedades de isotropia das camadas e substituir as coordenadas horizontais somente pelo raio.

A implementa\c{c}\~ao do algoritmo computacional ser\'a realizada em linguagem C++, por conta de algumas caracter\'isticas apresentadas por esta linguagem descritas em \cite{bueno_2015}, como: ser de prop\'osito geral podendo ser utilizada na constru\c{c}\~ao de programas computacionais, aplicativos de sistemas embarcados e em computa\c{c}\~ao cient\'ifica; ser de alto n\'ivel e orientada a objeto, permitindo a propagama\c{c}\~ao simult\^anea realizada por v\'arios programadores trabalhando num mesmo projeto; fortemente tipada o que ajuda na detec\c{c}\~ao de \textit{bugs} e controle e gerenciamento de mem\'oria; ser a mais utilizada em sistemas complexos e grandes no uso de programa\c{c}\~ao paralela.