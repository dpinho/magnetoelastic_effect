\chapter{Fundamenta\c{c}\~ao}
Segundo Farlow, a EDP de uma onda 
\begin{equation}\label{eq.edp_geral}
\frac{\partial^2\mathbf{f}(\mathbf{x},t)}{\partial\,t^2}=\norm{\mathbf{v}}^2\nabla^2\mathbf{f}(\mathbf{x},t)
\end{equation}
possui a solu\c{c}\~ao de D'Alembert
\begin{equation}
\mathbf{f}(\mathbf{x},t)=\mathbf{g}_1(\mathbf{x}-\mathbf{v}\,t)+\mathbf{g}_2(\mathbf{x}+\mathbf{v}\,t),
\end{equation}
onde $\mathbf{x}=(x,y,z)^\top$ representa o espa\c{c}o $\mathbb{R}^3$, $\mathbf{v}$ \'e a \textit{velocidade} de propaga\c{c}\~ao da onda, $(\mathbf{x}\pm\mathbf{v}\,t)$ \'e a \textit{fase} da onda, $\mathbf{g}_1$ \'e a propaga\c{c}\~ao da onda no semiespa\c{c}o positivo de $x$ e $\mathbf{g}_2$ \'e a propaga\c{c}\~ao da onda no semiespa\c{c}o negativo de $x$.

De acordo com FULANO, ondas tridimensionais se propagam em formato esferoidal mas localmente podem ser tratadas como ondas planas, DEFINIR ONDAS PLANAS principalmente para raios distantes da fonte, e assim a solu\c{c}\~ao de uma onda pode ser dada pela superposi\c{c}\~ao dessas ondas planas.

No $\mathbb{R}^3$ o \textit{vetor de onda} $\mathbf{k}=(k_x,k_y,k_z)$ \'eh aquele que aponta na dire\c{c}\~ao de propaga\c{c}\~ao da onda e sua magnitude, denominda \textit{n\'umero de onda}, \'e definida como 
\begin{equation}
\norm{\mathbf{k}}=k=\frac{\omega}{\norm{\mathbf{v}}},
\end{equation}
onde $\omega$ \'e a frequ\^encia temporal. Desta forma, a fase da onda pode ser escrita em termos do vetor de onda e da frequ\^encia como $(\mathbf{k}\cdot\mathbf{x}-\omega\,t)$, e a solu\c{c}\~ao da equa\c{c}\~ao \ref{eq.edp_geral} pode ser reescrita como uma superposi\c{c}\~ao de ondas planas
\begin{equation}
\mathbf{f}(\mathbf{x},t)=\mathbf{A}\,\sum_{\mathbf{k},\omega}{e^{i\,(\mathbf{k}\cdot\mathbf{x}-\omega\,t)}},
\end{equation}
onde $\mathbf{A}$ \'e a \textit{amplitude} da onda.



