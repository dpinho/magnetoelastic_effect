\chapter{Introdu\c{c}\~ao}
Como podemos constatar em \cite{eringen_1963}, existe uma similaridade matem\'atica entre a propaga\c{c}\~ao de ondas eletromagn\'eticas e el\'asticas em camadas que comp\~oem a subsuperf\'icie terrestre, que faz com que todas essas ondas tridimensionais possam ser representadas por equa\c{c}\~oes que possuem as mesmas propriedades. Verificamos tamb\'em que \'e poss\'ivel realizar o acoplamento dessas ondas, dentro de uma teoria conhecida como \textit{magnetoelasticidade}. Assim, segundo \cite{Ursin-1983}, podemos aplicar a mesma abordagem para tratamento tanto de ondas ac\'usticas como de ondas eletromagn\'eticas, e esse trabalho visa fazer o levantamento te\'orico de tal aplica\c{c}\~ao. A abordagem consiste basicamente em usar as tranformadas de Fourier, Laplace e Bessel no sistema de equa\c{c}\~oes diferenciais parciais que descrevem o acoplamento magnetoel\'astico, escrevendo-o em fun\c{c}\~ao da frequ\^encia temporal e numa forma matricial. Atrav\'es de transforma\c{c}\~ao das coordenadas laterais, o sistema \'e deixado em fun\c{c}\~ao apenas da profundidade e o sistema inical de EDP's \'e transformado em quatro sistemas de equa\c{c}\~oes diferenciais ordin\'arias onde cada grandeza pode ser calculada separadamente. Tal procedimento \'e denominado \textit{m\'etodo matricial}.

Alguns pesquisadores j\'a utilizaram o m\'etodo matricial em sistema acoplados. \cite{White_Zhou_2006} utilizaram em m\'etodos de \textit{Eletros\'ismica} que estuda a convers\~ao de ondas eletromagn\'eticas em ondas s\'ismicas na subsuperf\'icie terrestre na prospec\c{c}\~ao de hidrocarbonetos. A convers\~ao entre as ondas ocorre em meios porosos onde uma onda eletromagn\'etica pode excitar uma onda s\'ismica de mesma frequ\^encia e vice-versa, atrav\'es do movimento dos fluidos contidos nos poros. A altera\c{c}\~ao que uma onda eletromagn\'etica promove numa onda s\'ismica de mesma frequ\^encia pode ser registrada na superf\'icie terrestre trazendo informa\c{c}\~oes sobre as propriedades el\'etricas da subsuperf\'icie. 

\cite{Azeredo_2013} aplica o m\'etodo matricial para resolver de forma anal\'itico-num\'erica equa\c{c}\~oes de poroelasticidade que descrevem a propaga\c{c}\~ao de ondas em meio plano estratificado formado por camadas homog\^eneas e isotr\'opicas. O m\'etodo fornece f\'ormulas expl\'icitas para a constru\c{c}\~ao de algoritmo computacional  para obter a solu\c{c}\~ao do problema.