\chapter{Introdu\c{c}\~ao e Objetivos}

Como podemos constatar em \cite{eringen_1963}, existe uma similaridade matem\'atica entre a propaga\c{c}\~ao de ondas eletromagn\'eticas e el\'asticas em camadas que comp\~oem a subsuperf\'icie terrestre, que faz com que todas essas ondas tridimensionais possam ser representadas por equa\c{c}\~oes que possuem as mesmas propriedades. Verificamos tamb\'em que \'e poss\'ivel realizar o acoplamento dessas ondas, dentro de uma teoria conhecida como \textit{magneto-elasticidade}. Assim, segundo \cite{Ursin-1983}, podemos aplicar a mesma abordagem para tratamento tanto de ondas ac\'usticas como de ondas eletromagn\'eticas, e esse trabalho visa fazer o levantamento te\'orico de tal aplica\c{c}\~ao. 

Vamos estabelecer alguns fundamentos e tratar matematicamente o efeito magnetoel\'astico descrito em \cite{pinho_2018}. Para isso, precisamos como pre-requisito, do entendimento de v\'arias ferramentas da f\'isica-matem\'atica como a EDP de uma onda, mudan\c{c}a de sistemas de coordenadas, transformadas de Fourier, transformadas de Fourier-Bessel tamb\'em conhecidas como transformadas de Hankel, fun\c{c}\~ao Delta de Dirac e um m\'etodo matricial para solu\c{c}\~ao de EDO's.

A abordagem consiste basicamente em usar as tranformadas de Fourier no sistema de equa\c{c}\~oes diferenciais parciais que descrevem o acoplamento magneto-el\'astico, escrevendo-o em fun\c{c}\~ao da frequ\^encia temporal e em fun\c{c}\~ao da magnitude do vetor de ondas. Atrav\'es de transforma\c{c}\~ao das coordenadas laterais, o sistema \'e deixado em fun\c{c}\~ao da profundidade e o sistema inical de EDP's \'e transformado em quatro sistemas de equa\c{c}\~oes diferenciais ordin\'arias onde cada variável pode ser calculada separadamente. Com mudan\c{c}a no sistema de coordenadas e algum algebrismo, podemos simplificar as equa\c{c}\~oes e em seguida aplicamos o procedimento denominado \textit{m\'etodo matricial}, o qual nos permite obter a solu\c{c}\~ao das EDO's em cada camada de subsuperf\'icie. Retornamos as solu\c{c}\~oes para o sistema de coordenadas iniciais, aplicamos a transformada de Hankel com aux\'ilio das fun\c{c}\~oes de Bessel e por \'ultimo a transformada inversa de Fourier para obter as solu\c{c}\~oes no espa\c{c}o real novamente.

