\chapter{Vetor de Onda e Transformada de Fourier}
Segundo Farlow, a EDP de uma onda 
\begin{equation}\label{eq.edp_geral}
\frac{\partial^2\mathbf{f}(\mathbf{x},t)}{\partial\,t^2}=\norm{\mathbf{v}}^2\nabla^2\mathbf{f}(\mathbf{x},t)
\end{equation}
possui a solu\c{c}\~ao de D'Alembert
\begin{equation}
\mathbf{f}(\mathbf{x},t)=\mathbf{g}_1(\mathbf{x}-\mathbf{v}\,t)+\mathbf{g}_2(\mathbf{x}+\mathbf{v}\,t),
\end{equation}
onde $\mathbf{x}=(x,y,z)^\top$ representa o espa\c{c}o $\mathbb{R}^3$, $\mathbf{v}$ \'e a \textit{velocidade} de propaga\c{c}\~ao da onda, $(\mathbf{x}\pm\mathbf{v}\,t)$ \'e a \textit{fase} da onda, $\mathbf{g}_1$ \'e a propaga\c{c}\~ao da onda no semiespa\c{c}o positivo do eixo $x$ e $\mathbf{g}_2$ \'e a propaga\c{c}\~ao da onda no semiespa\c{c}o negativo do eixo $x$.

De acordo com FULANO, ondas tridimensionais se propagam em formato esferoidal mas localmente podem ser tratadas como ondas planas, DEFINIR ONDAS PLANAS principalmente para raios distantes da fonte, e assim a solu\c{c}\~ao de uma onda pode ser dada pela superposi\c{c}\~ao dessas ondas planas.

No $\mathbb{R}^3$ o \textit{vetor de onda} $\mathbf{k}=(k_x,k_y,k_z)$ \'e aquele que aponta na dire\c{c}\~ao de propaga\c{c}\~ao da onda e sua magnitude, denominada \textit{n\'umero de onda}, \'e definida como 
\begin{equation}
\norm{\mathbf{k}}=k=\frac{\omega}{\norm{\mathbf{v}}},
\end{equation}
onde $\omega$ \'e a frequ\^encia temporal. Desta forma, a fase da onda pode ser escrita em termos do vetor de onda e da frequ\^encia como $(\mathbf{k}\cdot\mathbf{x}-\omega\,t)$, e a solu\c{c}\~ao da equa\c{c}\~ao \ref{eq.edp_geral} pode ser reescrita como uma superposi\c{c}\~ao de ondas planas
\begin{equation}
\mathbf{f}(\mathbf{x},t)=\mathbf{A}\,\sum_{\mathbf{k},\omega}{e^{i\,(\mathbf{k}\cdot\mathbf{x}-\omega\,t)}},
\end{equation}
onde $\mathbf{A}$ \'e a \textit{amplitude} da onda.

Segundo \cite{White_Zhou_2006}, para o caso $\mathbb{R}^2$ o \textit{vetor de onda horizontal} \'e definido como $\mathbf{k}=(k_x,k_y)^\top$, o \textit{n\'umero de onda horizontal} e a \textit{vagarosidade horizontal} s\~ao, respectivamente,
\begin{equation}\label{eq.numero_onda_vagarozidade_horizontal}
k=\sqrt{k_x^2+k_y^2}\qquad\text{e}\qquad\gamma=\frac{k}{\omega}.
\end{equation}
A vagarosidade vertical \'e definida como
\begin{equation}
q_0=\frac{1}{v_z},
\end{equation}
onde $v_z$ \'e a componente vertical da velocidade. Denotando o \textit{n\'umero de onda vertical} por $k_z$, temos que a \textit{vagarosidade vertical} pode ser escrita como
\begin{equation}
q_0=\frac{k_z}{\omega}.
\end{equation}
Combinando as vagarosidades horizontal e vertical, temos
\begin{equation}\label{eq.vagarosidade_vertical}
\gamma^2+q_0^2=\frac{1}{\norm{\mathbf{v}}^2}\qquad\text{ou}\qquad q_0=\sqrt{\epsilon_0\mu_0-\gamma^2},
\end{equation}
j\'a que $\epsilon_0\mu_0=\frac{1}{\norm{\mathbf{v}}^2}$.
Podemos definir as transformadas de Fourier direta e inversa entre o espa\c{c}o bidimensional e o vetor de onda horizontal, como
\begin{align}\label{eq.trans_fourier_1}
\mathbf{\widehat{F}}(k_x,k_y,z) &= \iint_{\mathbb{R}^2}\mathbf{F}(x,y,z)\,e^{-i(k_xx+k_yy)}dx\,dy\\\nonumber\\\label{eq.trans_fourier_2}
\mathbf{F}(x,y,z) &= \left(\frac{1}{2\,\pi}\right)^2\iint_{\mathbb{R}^2}\mathbf{\widehat{F}}(k_x,k_y,z)\,e^{i(k_xx+k_yy)}dk_xdk_y.
\end{align}
O s\'imbolo $\,\widehat{.}\,$ denota a fun\c{c}\~ao no espa\c{c}o da transformada lateral de Fourier.
INCLUIR HIPOTESE DE CAMPOS TENDENDO A ZERO NO INFINITO
Segundo \cite{lang_1986}, podemos produzir uma rota\c{c}\~ao antihor\'aria em torno do eixo $z$ aplicando o operador linear
\begin{equation*}
\begin{pmatrix}
\cos\theta&-\sin\theta&0\\
\sin\theta&\cos\theta&0\\
0&0&1
\end{pmatrix},
\end{equation*}
onde $\theta$ \'e o \^angulo que um vetor est\'a sendo rotacionado. Como a geometria do nosso problema considera ondas se propagando na parte negativa do eixo $z$, para produzirmos uma rota\c{c}\~ao antihor\'aria devemos considerar o \^angulo $-\theta$, e com isso nossa matriz de rota\c{c}\~ao se torna
\begin{equation*}
\begin{pmatrix}
\cos\theta&\sin\theta&0\\
-\sin\theta&\cos\theta&0\\
0&0&1
\end{pmatrix},
\end{equation*}
lembrando a paridade das fun\c{c}\~oes seno e cosseno. Para promovermos uma rota\c{c}\~ao orientando a primeira coordenada  no sentido de propaga\c{c}\~ao das ondas horizontais, temos que $\theta$ ser\'a o \^angulo entre $(x,0,0)^\top$ e $(k_x,k_y,0)^\top$, e a matriz de rota\c{c}\~ao se torna
\begin{equation}\label{eq.operador_rotacao}
\Omega=
\begin{pmatrix}
\frac{k_x}{k}&\frac{k_y}{k}&0\\
-\frac{k_y}{k}&\frac{k_x}{k}&0\\
0&0&1
\end{pmatrix}.
\end{equation}