\chapter{M\'etodo Matricial de Ursin para Solu\c{c}\~ao de EDP's}
Este cap\'itulo trata da utiliza\c{c}\~ao de um m\'etodo matricial para estudar a propaga\c{c}\~ao de ondas em subsuperf\'icie terrestre, conforme estruturado em \cite{Ursin-1983}. O m\'etodo utiliza um conjunto de transformadas para escrever um sistema de EDP's em dois sistemas de EDO's em forma matricial, de modo que se possa separar o campo de ondas em ascendentes e descendentes atrav\'es de uma decomposi\c{c}\~ao em autovetores. \'E poss\'ivel computar a matriz de propaga\c{c}\~ao das ondas e as matrizes de transmiss\~ao e reflex\~ao de ondas nas fronteiras entre camadas atrav\'es de propriedades de simetrias.

\section{Escrevendo as Equa\c{c}\~oes na Forma Matricial}

Sendo $\mathbf{x}=(x_1,x_2,x_3)^{\top}$ o espa\c{c}o $\mathbb{R}^3$ e aplicando as tranformadas de Fourier direta e inversa na forma
\begin{align*}
F(\omega,k_1,k_2,x_3) &= \iiint_{-\infty}^{\infty}f(t,x_1,x_2,x_3)\,e^{i\omega t-ik_1x_1-ik_2x_2}dt\,dx_1dx_2\\\\
f(t,x_1,x_2,x_3) &= \left(\frac{1}{2\,\pi}\right)^3\,\iiint_{-\infty}^{\infty}F(\omega,k_1,k_2,x_3)\,e^{-i\omega t+ik_1x_1+ik_2x_2}d\omega\,dk_1dk_2\,,
\end{align*}
podemos escrever um conjunto de EDP's que descevem a propaga\c{c}\~ao de ondas sismomagn\'eticas em camadas horizontais da subsuperf\'icie terrestre somente em fun\c{c}\~ao da profundidade $x_3$. 

A t\'itulo de exemplo, tanto as EDP's de Maxwell para o eletromagnetismo
\begin{align}\label{eq.faraday_ampere}\nonumber
\nabla\times\mathbf{E}&=-\frac{\partial}{\partial t}\mathbf{B}\\\\\nonumber
\nabla\times\mathbf{H}&=\sigma\mathbf{E}+\frac{\partial}{\partial t}\mathbf{D}+\mathbf{G}\,,
\end{align}
como as EDP's el\'asticas
\begin{align}\label{eq.cauchy_hooke}\nonumber
\rho\frac{\partial^2 \mathbf{U}}{\partial t^2}&=\nabla\cdot\tau+\mathbf{F}\\\\\nonumber
\tau&=\lambda\nabla\cdot \mathbf{U}\cdot I + \mu(\nabla \mathbf{U}+\nabla \mathbf{U}^*)\,,
\end{align}
podem ser escritas no formato matricial apresentado por Ursin. Cada um desses sistemas, isoladamente, podem ser escritos no formato 
\begin{align}\label{eq.matricial}
\frac{\partial\,\mathbf{B}}{\partial\,z} &= \pm\,i\,\omega\,A\,\mathbf{B} = \pm\,i\,\omega\,
\begin{bmatrix}
0&A_1\\
A_2&0
\end{bmatrix}\,
\begin{bmatrix}
\mathbf{B_1}\\
\mathbf{B_2}	
\end{bmatrix}\,,
\end{align}
onde o vetor $\mathbf{B}$ representa uma onda qualquer e n\~ao o campo magn\'etico dado em \ref{eq.faraday_ampere}.

A equação \ref{eq.matricial} tem as seguintes caracter\'isticas:
\begin{itemize}
\item $A_{2\,n\times2\,n}$ \'e uma matriz que pode ser particionada em quatro submatrizes $n\times n$, com submatrizes de zeros na diagonal principal e submatrizes sim\'etricas $A_1$ e $A_2$ na diagonal secund\'aria. As componentes de $A_1$ e $A_2$ s\~ao fun\c{c}\~oes dos par\^ametros das EDP's \ref{eq.faraday_ampere} e \ref{eq.cauchy_hooke}, s\~ao fun\c{c}\~oes de $x_3$ e tamb\'em do vetor real de retardamento $\mathbf{p}=\frac{\mathbf{k}}{\omega}$. Para meios de baixa atenua\c{c}\~ao das ondas, as matrizes $A_1$ e $A_2$ s\~ao reais; 
\item O vetor de onda $\mathbf{B}$ tem dimens\~ao $2n\times1$ e \'e particionado em dois vetores $\mathbf{B}_1$ e $\mathbf{B}_2$ com dimens\~ao $n\times1$ . As componentes do vetor de onda s\~ao escolhidas de forma que $\mathbf{B}$ seja cont\'inuo atrav\'es das fronteiras entre duas camadas;
\item  Para ondas el\'asticas, metade das componentes de $\mathbf{B}$ s\~ao zeros na superf\'icie livre, ou seja, existe uma matriz de permuta\c{c}\~ao $T_{2n\times2n}$ onde $T^{-1}=T^\top$ e tal que
\begin{equation}
\begin{bmatrix}
\mathbf{V}_1(\mathbf{0})\\
\mathbf{0}
\end{bmatrix}
=T\,\mathbf{B}\quad\text{quando}\quad x_3 = 0\,;
\end{equation}
\item As componentes do vetor de onda $\mathbf{B}$ s\~ao escolhidas de forma que o fluxo de energia na dire\c{c}\~ao $x_3$ seja dado por
\begin{equation}
J=-\frac{1}{4}(\mathbf{B}_1^H\mathbf{B}_2+\mathbf{B}_2^H\mathbf{B}_1)=-\frac{1}{4}\mathbf{B}^H\,M\, \mathbf{B}\,,
\end{equation}
onde $H$ denota complexo conjugado transposto,
\begin{equation}
M=
\begin{bmatrix}
0_{n\times n}&I\\
I&0_{n\times n}
\end{bmatrix}
\end{equation}
e $I$ \'e uma matriz identidade $n\times n$.
\end{itemize}

Os metodos a seguir sao aplicados em equacoes escritas no formato matricial \ref{eq.matricial}, com ondas se propagando numa pilha de camadas nao homogeneas e assumimos que os parametros das equacoes sao funcoes continuas no interior de cada camada que dependem apenas da profunidade $x_3$. O modelo inclui pilha de camadas homogeneas com parametros constantes por camada e consideramos o eixo $z$ como sendo positivo no sentido descendente.   

\section{Decomposicao em Ondas Ascendentes e Descentes}

Para realizar a decommposicao do vetor $\mathbf{B}$ em ondas ascendentes e descendentes aplicamos uma diagonalizacao em autovalores na matriz $A$ na forma
\begin{equation}
A=L\,A_1L^{-1}
\end{equation}



