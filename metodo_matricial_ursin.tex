\chapter{M\'etodo Matricial de Ursin para Solu\c{c}\~ao de EDP's}
Este cap\'itulo trata da utiliza\c{c}\~ao de um m\'etodo matricial para estudar a propaga\c{c}\~ao de ondas em subsuperf\'icie terrestre, conforme estruturado em \cite{Ursin-1983}. O m\'etodo utiliza um conjunto de transformadas para escrever um sistema de EDP's em dois sistemas de EDO's em forma matricial, de modo que se possa separar o campo de ondas em ascendentes e descendentes atrav\'es de uma decomposi\c{c}\~ao em autovetores. \'E poss\'ivel computar a matriz de propaga\c{c}\~ao das ondas e as matrizes de transmiss\~ao e reflex\~ao de ondas nas fronteiras entre camadas atrav\'es de propriedades de simetrias.

\section{Escrevendo as Equa\c{c}\~oes na Forma Matricial}

Sendo $\mathbf{x}=(x_1,x_2,x_3)^{\top}$ o espa\c{c}o $\mathbb{R}^3$ e aplicando as tranformadas de Fourier direta e inversa na forma
\begin{align*}
F(\omega,k_1,k_2,x_3) &= \iiint_{-\infty}^{\infty}f(t,x_1,x_2,x_3)\,e^{i\omega t-ik_1x_1-ik_2x_2}dt\,dx_1dx_2\\\\
f(t,x_1,x_2,x_3) &= \left(\frac{1}{2\,\pi}\right)^3\,\iiint_{-\infty}^{\infty}F(\omega,k_1,k_2,x_3)\,e^{-i\omega t+ik_1x_1+ik_2x_2}d\omega\,dk_1dk_2\,,
\end{align*}
podemos escrever um conjunto de EDP's que descevem a propaga\c{c}\~ao de ondas sismomagn\'eticas em camadas horizontais da subsuperf\'icie terrestre somente em fun\c{c}\~ao da profundidade $x_3$. 

A t\'itulo de exemplo, tanto as EDP's de Maxwell para o eletromagnetismo
\begin{align}\label{eq.faraday_ampere}\nonumber
\nabla\times\mathbf{E}&=-\frac{\partial}{\partial t}\mathbf{B}\\\\\nonumber
\nabla\times\mathbf{H}&=\sigma\mathbf{E}+\frac{\partial}{\partial t}\mathbf{D}+\mathbf{G}\,,
\end{align}
como as EDP's el\'asticas
\begin{align}\label{eq.cauchy_hooke}\nonumber
\rho\frac{\partial^2 \mathbf{U}}{\partial t^2}&=\nabla\cdot\tau+\mathbf{F}\\\\\nonumber
\tau&=\lambda\nabla\cdot \mathbf{U}\cdot I + \mu(\nabla \mathbf{U}+\nabla \mathbf{U}^*)\,,
\end{align}
podem ser escritas no formato matricial apresentado por Ursin. Cada um desses sistemas, isoladamente, pode ser escrito no formato 
\begin{align}\label{eq.matricial}
\frac{\partial\,\mathbf{B}}{\partial\,z} &= \pm\,i\,\omega\,A\,\mathbf{B} = \pm\,i\,\omega\,
\begin{bmatrix}
0&A_1\\
A_2&0
\end{bmatrix}\,
\begin{bmatrix}
\mathbf{B_1}\\
\mathbf{B_2}	
\end{bmatrix}\,,
\end{align}
onde o vetor $\mathbf{B}$ representa uma onda qualquer e n\~ao o campo magn\'etico dado em \ref{eq.faraday_ampere}.

A equação \ref{eq.matricial} tem as seguintes caracter\'isticas:
\begin{itemize}
\item $A_{2\,n\times2\,n}$ \'e uma matriz que pode ser particionada em quatro submatrizes $n\times n$, com submatrizes de zeros na diagonal principal e submatrizes sim\'etricas $A_1$ e $A_2$ na diagonal secund\'aria. As componentes de $A_1$ e $A_2$ s\~ao fun\c{c}\~oes dos par\^ametros das EDP's \ref{eq.faraday_ampere} e \ref{eq.cauchy_hooke}, s\~ao fun\c{c}\~oes de $x_3$ e tamb\'em do vetor real de retardamento $\mathbf{p}=\frac{\mathbf{k}}{\omega}$. Para meios de baixa atenua\c{c}\~ao das ondas, as matrizes $A_1$ e $A_2$ s\~ao reais; 
\item O vetor de onda $\mathbf{B}$ tem dimens\~ao $2n\times1$ e \'e particionado em dois vetores $\mathbf{B}_1$ e $\mathbf{B}_2$ com dimens\~ao $n\times1$ . As componentes do vetor de onda s\~ao escolhidas de forma que $\mathbf{B}$ seja cont\'inuo atrav\'es das fronteiras entre duas camadas;
\item  Para ondas el\'asticas, metade das componentes de $\mathbf{B}$ s\~ao zeros na superf\'icie livre, ou seja, existe uma matriz de permuta\c{c}\~ao $T_{2n\times2n}$ onde $T^{-1}=T^\top$ e tal que
\begin{equation*}
\begin{bmatrix}
\mathbf{V}_1(\mathbf{0})\\
\mathbf{0}
\end{bmatrix}
=T\,\mathbf{B}\quad\text{quando}\quad x_3 = 0\,;
\end{equation*}
\item As componentes do vetor de onda $\mathbf{B}$ s\~ao escolhidas de forma que o fluxo de energia na dire\c{c}\~ao $x_3$ seja dado por
\begin{equation*}
J=-\frac{1}{4}(\mathbf{B}_1^H\mathbf{B}_2+\mathbf{B}_2^H\mathbf{B}_1)=-\frac{1}{4}\mathbf{B}^H\,M\, \mathbf{B}\,,
\end{equation*}
onde $H$ denota complexo conjugado transposto,
\begin{equation*}
M=
\begin{bmatrix}
0_{n\times n}&I\\
I&0_{n\times n}
\end{bmatrix}
\end{equation*}
e $I$ \'e uma matriz identidade $n\times n$.
\end{itemize}

Os m\'etodos a seguir s\~ao aplicados em equa\c{c}\~oes escritas no formato matricial \ref{eq.matricial}, com ondas se propagando numa pilha de camadas n\~ao homog\^eneas e assumimos que os par\^ametros das equa\c{c}\~oes s\~ao fun\c{c}\~oes cont\'inuas no interior de cada camada que dependem apenas da profundidade $x_3$. O modelo inclui pilha de camadas homog\^eneas com par\^ametros constantes por camada e consideramos o eixo $z$ como sendo positivo no sentido descendente.   

\section{Decomposi\c{c}\~ao em Ondas Ascendentes e Descentes}

Para realizar a decomposi\c{c}\~ao do vetor $\mathbf{B}$ em ondas ascendentes e descendentes aplicamos uma diagonaliza\c{c}\~ao em autovalores na matriz $A$ na forma
\begin{equation}\label{eq.diagonalizacao}
A=L\,\Lambda_1L^{-1}\,,
\end{equation}
onde $\Lambda_1$ \'e a matriz diagonal dos autovalores $\lambda_i$ para $i=1,2,...,n$, e $L$ \'e a matriz dos autovetores correspondentes,
\begin{equation*}
\Lambda_1=
\begin{bmatrix}
\Lambda&0\\
0&-\Lambda
\end{bmatrix}\,.
\end{equation*} 
A defini\c{c}\~ao de autovalores e autovetores \'e dada por
\begin{equation}\label{eq.sist_autovalores}
\begin{bmatrix}
0&A_1\\
A_2&0
\end{bmatrix}
\begin{bmatrix}
\mathbf{L_1}\\
\mathbf{L_2}
\end{bmatrix}
=
\lambda\,
\begin{bmatrix}
\mathbf{L_1}\\
\mathbf{L_2}
\end{bmatrix}
\end{equation}
e aplicando o procedimento
\begin{equation*}
\begin{bmatrix}
0&A_1\\
A_2&0
\end{bmatrix}
\begin{bmatrix}
0&A_1\\
A_2&0
\end{bmatrix}
\begin{bmatrix}
\mathbf{L_1}\\
\mathbf{L_2}
\end{bmatrix}
=
\lambda\,\lambda\,
\begin{bmatrix}
\mathbf{L_1}\\
\mathbf{L_2}
\end{bmatrix}\,,
\end{equation*}
podemos separar o sitema \ref{eq.sist_autovalores} de dimens\~ao $2n$ em dois sistemas de dimens\~ao n
\begin{align*}
A_1A_2\mathbf{L_1}&=\lambda^2\mathbf{L_1}\\
A_2A_1\mathbf{L_2}&=\lambda^2\mathbf{L_2}\,,
\end{align*}
onde $\mathbf{L}_i$ s\~ao os autovetores que comp\~oem $L$.
Assim, podemos dividir a diagonaliza\c{c}\~ao dada em \ref{eq.diagonalizacao} em duas diagonaliza\c{c}\~oes de dimens\~ao $n$
\begin{align}\label{eq.subdiagonalizacao}\nonumber
A_1A_2&=L_1\Lambda^2L_1^{-1}\\\quad\\\nonumber
A_2A_1&=L_2\Lambda^2L_2^{-1}\,,
\end{align}
onde $L_i$ s\~ao submatrizes da matriz $L$ e cont\^em os autovetores $\mathbf{L}_i$.



Definindo a matriz 
\begin{equation}\label{eq.L}
L=\frac{1}{\sqrt{2}}
\begin{bmatrix}
L_1&L_1\\
L_2&-L_2
\end{bmatrix}
\end{equation}
e sua inversa
\begin{equation}\label{eq.L_inversa}
L^{-1}=\frac{1}{\sqrt{2}}
\begin{bmatrix}
L_1^{-1}&L_2^{-1}\\
L_1^{-1}&-L_2^{-1}
\end{bmatrix}\,,
\end{equation}
podemos substitui-las na equa\c{c}\~ao \ref{eq.diagonalizacao} e verificar que 
\begin{align}\label{eq.definicao_A1_A2}\nonumber
A_1&=L_1\Lambda\,L_2^{-1}\\\quad\\\nonumber
A_2&=L_2\Lambda\,L_1^{-1}\,,
\end{align}
e podemos verificar ainda que essas defini\c{c}\~oes para $A_1$ e $A_2$ satisfazem tamb\'em as equa\c{c}\~oes \ref{eq.subdiagonalizacao}.

Pelas caracter\'isticas da equa\c{c}\~ao \ref{eq.matricial} sabemos que as matrizes $A_1$ e $A_2$ s\~ao sim\'etricas e podem ser escritas como
\begin{align*}
A_1&=L_2^{-\top}\Lambda\,L_1^\top\\
A_2&=L_1^{-\top}\Lambda\,L_2^\top\,,
\end{align*}
as quais substitu\'idas nas equa\c{c}\~oes \ref{eq.subdiagonalizacao} lucramos
\begin{equation*}
A_1A_2=L_1\Lambda^2L_1^{-1}=L_2^{-\top}\Lambda^2L_2^\top\,,
\end{equation*}
e conclu\'imos que, a menos da escala dos autovetores,
\begin{equation*}
L_1=L_2^{-\top}\,.
\end{equation*}
Substituindo a \'ultima igualdade na defini\c{c}\~ao \ref{eq.definicao_A1_A2} temos
\begin{equation*}
A_i=L_i\Lambda\,L_i^{\top}\qquad\text{para}\qquad i=1,2\,,
\end{equation*}
e substituindo em \ref{eq.L_inversa}, temos
\begin{equation}\label{eq.L_inversa_2}
L^{-1}=\frac{1}{\sqrt{2}}
\begin{bmatrix}
L_2^{\top}&L_1^{\top}\\
L_2^{\top}&-L_1^{\top}
\end{bmatrix}\,.
\end{equation}

Escrevendo o vetor de ondas na forma
\begin{equation*}
\mathbf{B}=\mathbf{L}\,\mathbf{W}\,,
\end{equation*}
aplicando a derivada parcial em rela\c{c}\~ao a $x_3$, e substituindo as equa\c{c}\~oes \ref{eq.matricial} e \ref{eq.diagonalizacao}, obtemos
\begin{equation}\label{eq.derivada_W}
\frac{\partial\mathbf{W}}{\partial\,x_3}=\left[\pm i\omega\,\Lambda_1-L^{-1}\frac{\partial\,L}{\partial\,x_3}\right]\,\mathbf{W}\,.
\end{equation}
Para a propaga\c{c}\~ao de ondas em camadas homog\^eneas, os coeficientes das EDP's originais s\~ao constantes por camada e esses coeficientes comp\~oem a matriz $A$ diagonalizada pela matriz $L$. Assim, para meios homog\^eneos, a \'ultima equa\c{c}\~ao pode ser reduzida a
\begin{equation*}
\frac{\partial\mathbf{W}}{\partial\,x_3}=\pm i\omega\,\Lambda_1\mathbf{W}\,.
\end{equation*}
Representamos o vetor $\mathbf{W}$ como
\begin{equation*}
\mathbf{W}=
\begin{bmatrix}
\mathbf{U}\\
\mathbf{D}
\end{bmatrix}\,,
\end{equation*}
onde $\mathbf{U}$ e $\mathbf{D}$ s\~ao vetores que representam ondas ascendentes e descendentes, respectivamente, desde que a parte real de $\pm i\omega\lambda_i$ seja n\~ao negativa para $i=1,2,...,n$.

Substiutindo as equa\c{c}\~oes \ref{eq.L} e \ref{eq.L_inversa_2} na equa\c{c}\~ao \ref{eq.derivada_W} e efetuando as multiplica\c{c}\~oes matriciais obtemos
\begin{align}\label{eq.Up_Down}\nonumber
\frac{\partial\mathbf{U}}{\partial x_3}&=\pm i\omega\,\Lambda\,\mathbf{U}+F\,\mathbf{U}+G\,\mathbf{D}\\\quad\\\nonumber
\frac{\partial\mathbf{D}}{\partial x_3}&=\pm i\omega\,\Lambda\,\mathbf{D}+F\,\mathbf{D}+G\,\mathbf{U}
\end{align}
onde as matrizes $F$ e $G$ s\~ao, respectivamente,
\begin{align*}
F&=-\frac{1}{2}\left[L_2^\top\frac{\partial L_1}{\partial x_3}+L_1^\top\frac{\partial L_2}{\partial x_3}\right]\\\quad\\
G&=-\frac{1}{2}\left[L_2^\top\frac{\partial L_1}{\partial x_3}-L_1^\top\frac{\partial L_2}{\partial x_3}\right]\,.
\end{align*}
Substituindo $F$ e $G$ nas rela\c{c}\~oes
\begin{equation*}
-2(F+F^\top)\quad\text{e}\quad-2(G-G^\top)\,,
\end{equation*}
verificamos que ambas s\~ao nulas e, consequentemente,
\begin{equation*}
F=-F^\top\quad\text{e}\quad G=G^\top\,.
\end{equation*}
Negligenciando a \'ultima parcela das equa\c{c}\~oes \ref{eq.Up_Down}, temos a express\~ao final para ondas ascendentes e descendentes, respectivamente,
\begin{align}\label{eq.Up_Down_2}\nonumber
\frac{\partial\mathbf{U}}{\partial x_3}&=\pm i\omega\,\Lambda\,\mathbf{U}+F\,\mathbf{U}\\\quad\\\nonumber
\frac{\partial\mathbf{D}}{\partial x_3}&=\pm i\omega\,\Lambda\,\mathbf{D}+F\,\mathbf{D}\,.
\end{align}



