\chapter{Fundamentos de Elasticidade}\label{sec.fund_elast}

\section{Introdu\c{c}\~ao}
A teoria formal da propaga\c{c}\~ao de ondas s\'ismicas repousa nas intera\c{c}\~oes entre as particulas infinitesimais discretas do meio \`a medida que uma deforma\c{c}\~ao se propaga. \'E muito dif\'icil estudar individualmente cada uma dessas intera\c{c}\~oes, mas dados experimentais que foram coletados como resultados dessas intera\c{c}\~oes sugerem que as mesmas podem ser consideradas em conjunto. Assim, o estudo da propaga\c{c}\~ao de ondas s\'ismicas atrav\'es de camadas de subsuperf\'icie num material discretizado pode ser feito considerando o meio como cont\'inuo, e tais estudos s\~ao os objetos da \textit{mec\^anica do cont\'inuo}. 

No desenvolvimento te\'orico da mec\^anica do cont\'inuo nao s\~ao consideradas as caracter\'isticas at\^omicas da mat\'eria bem como as intera\c{c}\~oes entre essas part~\'iculas, ou seja, a mat\'eria n\~ao \'e estudada do ponto de vista microsc\'opico. Segundo \cite{slawinski}, tal abordagem se justifica pelo fato de que a mat\'eria \'e formada por part\'iculas suficientemente pouco espa\c{c}adas e suas caracter\'isticas e comportamento podem ser descritos por fun\c{c}\~oes cont\'inuas e diferenci\'aveis. Assim, \'e assummido que elementos infinitesimais da mat\'eria t\^em as mesmas propriedades observadas em experimentos macrosc\'opicos, pois essa hip\'otese permite a cria\c{c}\~ao de um modelo matem\'atico abstrato \textit{efetivo} na descri\c{c}\~ao da realidade f\'isica. Como exemplo, vamos considerar a cor de um objeto. Pr\'otons e el\'etrons n\~ao possuem cor, mas os meios materiais (que s\~ao formados por pr\'otons e el\'etrons) t\^em a capacidade de absorver ou refletir determinados comprimentos de ondas eletromagn\'eticas as quais determinam a cor de cada meio. Outros conceitos da mec\^anica do cont\'inuo sao elasticidade, viscosidade, fric\c{c}\~ao, rigidez, etc, como veremos mais a frente.


\section{Fatos experimentais}
A teoria sobre elasticidade est\'a baseada em conceitos primitivos e conclus\~oes estabelecidas a partir de fatos experimentais verificados em v\'arios textos sobre o assunto como \cite{liu}, \cite{dahlem} e \cite{slawinski}. Adicionalmente, em geral as equa\c{c}\~oes que governam a propaga\c{c}\~ao de ondas em meios el\'asticos s\~ao n\~ao-lineares. Contudo, em experimentos s\'ismicos foi constatado que aspectos importantes da propaga\c{c}\~ao de ondas podem ser analisados a partir de equa\c{c}\~oes lineares, resultando numa abordagem chamada \textit{teoria da elasticidade linearizada}.

\subsection{Deforma\c{c}\~ao}

A \textit{deforma\c{c}\~ao} de um meio el\'astico cont\'inuo \'e a mudan\c{c}a na posi\c{c}\~ao dos pontos que comp\~oem o corpo em rela\c{c}\~ao uns aos outros. Ou seja, h\'a uma mudan\c{c}a relativa entre os pontos e n\~ao um deslocamento do corpo como um todo e sem mudan\c{c}a de sua forma, caso em que ter\'iamos um \textit{movimento r\'igido}. Nesta subse\c{c}\~ao estamos interessados nas caracter\'isticas geom\'etricas relativas a deforma\c{c}\~ao de um corpo. N\~ao estamos considerando as causas de deforma\c{c}\~ao de um corpo, como aplica\c{c}\~ao de carga ou varia\c{c}\~ao de temperatura, nem discutiremos a composi\c{c}\~ao do material, assumindo apenas que o mesmo seja cont\'inuo e el\'astico. Assim, vamos relacionar as caracter\'isticas geom\'etricas de um corpo antes da deforma\c{c}\~ao com as caracter\'isticas ap\'os a deforma\c{c}\~ao.

\subsection{Dedu\c{c}\~ao do Tensor de Deforma\c{c}\~oes}

Para determinar o tensor de deforma\c{c}\~oes vamos considerar dois pontos pertencentes ao espaco $\mathbb{R}^3$ bastante pr\'oximos um do outro denotados por $\mathbf{x}=(x_1,x_2,x_3)$ e $\mathbf{y}=\mathbf{x}+d\mathbf{s}=(x_1+dx_1,x_2+dx_2,x_3+dx_3)$. O quadrado da dist\^ancia entre esses dois pontos \'e
\begin{equation*}
\norm{d\mathbf{s}}^2=(dx_1)^2+(dx_2)^2+(dx_3)^2.
\end{equation*}
Ap\'os a aplica\c{c}\~ao de uma deforma\c{c}\~ao do meio podemos definir o \textit{vetor de deslocamento}
\begin{equation*}
\mathbf{u}=(u_1,u_2,u_3)
\end{equation*}
e som\'a-lo aos pontos $\mathbf{x}$ e $\mathbf{y}$ para obter suas posi\c{c}\~oes ap\'os a deforma\c{c}\~ao,
\begin{align*}
\mathbf{x}^*&=(x_1+u_1,x_2+u_2,x_3+u_3)\\
\mathbf{y}^*&=(x_1+dx_1+u_1,x_2+dx_2+u_2,x_3+dx_3+u_3).
\end{align*}




\section{Equações de Lamé}

\section{Generalizações da teoria}

\section{Conclusões}