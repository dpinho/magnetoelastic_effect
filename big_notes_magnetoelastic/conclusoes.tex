\chapter{Conclusões e Trabalhos Futuros}

Muitas pesquisas v\^em sendo realizadas no sentido de efetuar simula\c{c}\~oes num\'erico-computacionais que possam descrever diversos fen\^omenos f\'isicos relacionados \`a prospec\c{c}\~ao de petr\'oleo ou outro bem mineral, assim como fen\^omenos f\'isicos relacionados a terremotos ou que se aplicam a outros objetos de estudo. Essas simula\c{c}\~oes s\~ao ainda confrontadas com experimentos de campo na busca por consist\^encia entre essas duas faces do desenvolvimento de uma teoria. Numa oportunidade futura vamos desenvolver de forma anal\'itico-matem\'atica as EDP's da magneto-elasticidade,  e em seguida criar um algoritmo computacional capaz de efetuar simula\c{c}\~oes que nos auxiliem a estudar o efeito magneto-el\'astico, bem como entender e expandir a teoria geral que trata da intera\c{c}\~ao entre mec\^anica do cont\'inuo e eletromagnetismo em explora\c{c}\~ao de petr\'oleo.

O efeito magneto-el\'astico \'e descrito matematicamente pelo conjunto de EDP's formado pelo sistema de Maxwell e pelo sistema de Lam\`e, os quais s\~ao utilizados no estudo da propaga\c{c}\~ao acoplada de ondas s\'ismicas e eletromagn\'eticas na subsuperf\'icie terrestre. O acoplamento foi caracterizado, primeiramente, pela varia\c{c}\~ao que a for\c{c}a de Lorentz provoca no deslocamento do meio condutivo, simbolizado pela adi\c{c}\~ao da parcela referente \`a esta for\c{c}a na equa\c{c}\~ao do movimento de Cauchy. Segundo, pela a altera\c{c}\~ao eletromagn\'etica gerada pela passagem de uma onda s\'ismica que faz um meio condutivo oscilar no campo geomagn\'etico, simbolizada pela adi\c{c}\~ao desta varia\c{c}\~ao \`a lei de Amp\`ere-Maxwell. Estamos estudando o caso parcialmente acoplado no espaco 3D, mas desejamos analisar tamb\'em o acoplamento total considerando tanto o espa\c{c}o 3D como o espa\c{c}o 1D, esperando que os desenvolvimentos e resultados em cada caso possam se complementar mutuamente.

Algumas hip\'oteses de ordem f\'isica, como o regime quasi-estacion\'ario por exemplo, foram necess\'arias para simplificar o modelo, linearizando as equa\c{c}\~oes e possibilitando o desenvolvimento anal\'itico das mesmas. Sendo assim, buscaremos pelas solu\c{c}\~oes dessas EDP's raciocinando basicamente com duas alternativas. Uma delas \'e trasnsformar as EDP's em EDO's utilizando ferramentas como as transformadas laterais de Fourier, transformadas de Hankel e mudan\c{c}as de eixos coordenados, escrevendo as equa\c{c}\~oes num formato onde \'e poss\'ivel aplicar um metodo matricial espec\'ifico para estudo de propaga\c{c}\~ao de ondas em meios estratigr\'aficos. Outra alternativa \'e reescrever as equa\c{c}\~oes em coordenadas cil\'indricas e fazer uso da hip\'otese de isotropia do meio de propaga\c{c}\~ao para reduzir as dimens\~oes do problema e obter as EDO's nas quais o m\'etodo matricial \'e aplicado.
\\
Neste trabalho apresentamos um tratamento matem\'atico das EDP's do efeito magneto-el\'astico encontrado em \cite{pinho_2018} , no sentido de propiciar a contru\c{c}\~ao de um algoritmo num\'erico est\'avel que possa descrever a propaga\c{c}\~ao acoplada de ondas el\'asticas e eletromagn\'eticas. Nesse tratamento foi fundamental a aplica\c{c}\~ao de conhecimentos da F\'isica-Matem\'atica, Geof\'isica e, em particular, um metodo matricial que facilita a an\'alise de propaga\c{c}\~ao de ondas em meios estratificados.

Vimos na subse\c{c}\~ao \ref{sec.matricial_poroelast} a possibilidade de an\'alise de dispers\~ao e de atenua\c{c}\~ao de ondas para casos diversos, onde tal an\'alise auxilia na verifica\c{c}\~ao e constru\c{c}\~ao de um c\'odigo computacional efetivo para descrever a propaga\c{c}\~ao dessas ondas. Numa oportuinidade futura, queremos aplicar a an\'alise de atenua\c{c}\~ao e dispers\~ao nesse sistema de EDP's do efeito magneto-el\'astico com a finalidade de ajudar a estudar o comportamento da propaga\c{c}\~ao.

Numa determinada abordagem, a an\'alise de casos mais simples auxilia no estudo de casos mais sofisticados. Por tanto, no intuito ainda de otimizar o estudo da propaga\c{c}\~ao das ondas, faremos o tratamento matem\'atico das EDP's de magneto-elasticidade para o caso unidimensional, considerando a propaga\c{c}\~ao em fun\c{c}\~ao do tempo e em fun\c{c}\~ao da profundidade. Neste caso podemos utilizar o m\'etodo matricial e a an\'alise de atenua\c{c}\~ao e dispers\~ao das ondas, e economizamos a utiliza\c{c}\~ao de transformadas e mudan\c{c}a de eixos coordenados.

O formato final das EDO's dado no cap\'itulo \ref{sec.trans_edp_2_edo} apresentou algumas vari\'aveis incluidas como fonte, diferentemente do que \'e preconizado por Ursin, onde todas a vari\'aveis devem estar inseridas no vetor $\mathbf{\Phi}$. Assim, analisaremos a possibilidade da aplica\c{c}\~ao de fun\c{c}\~oes de Green juntamente com o m\'etodo matricial para contornar esse problema. \'E poss\'ivel que essa abordagem traga desafios computacionais consider\'aveis e da\'i estudaremos tamb\'em outras alternativas. Uma delas \'e considerar o efeito magento-el\'astico para o caso totalmente acoplado e verificar se o novo formato das equa\c{c}\~oes permite a exclus\~ao de vari\'avies dadas como fonte. Outra possibilidade \'e escrever as equa\c{c}\~oes em coordenadas cil\'indricas, considerar as propriedades de isotropia das camadas e substituir as coordenadas horizontais somente pelo raio.

A implementa\c{c}\~ao do algoritmo computacional ser\'a realizada em linguagem C++, por conta de algumas caracter\'isticas apresentadas por esta linguagem descritas em \cite{bueno_2015}, como: ser de prop\'osito geral podendo ser utilizada na constru\c{c}\~ao de programas computacionais, aplicativos de sistemas embarcados e em computa\c{c}\~ao cient\'ifica; ser de alto n\'ivel e orientada a objeto, permitindo a propagama\c{c}\~ao simult\^anea realizada por v\'arios programadores trabalhando num mesmo projeto; fortemente tipada o que ajuda na detec\c{c}\~ao de \textit{bugs} e controle e gerenciamento de mem\'oria; ser a mais utilizada em sistemas complexos e grandes no uso de programa\c{c}\~ao paralela.