\chapter{Sistema de EDO's do Efeito Magneto-El\'astico}\label{sec.trans_edp_2_edo}
Neste cap\'itulo vamos aplicar algumas t\'ecnicas como rota\c{c}\~ao do sistema de coordenadas e Transformadas Laterais de Fourier em EDP's da magneto-elasticidade para que as mesmas possam ser escritas como um sistema de EDO's.

\section{Transformando EDP's em EDO's}

Segundo \cite{eringen_1963}, o acoplamento entre ondas eletromagn\'eticas e el\'asticas se propagando no subsolo caracteriza o efeito magneto-el\'astico, e esse acoplamento pode ser modelado matematicamente atrav\'es de um sistema de equa\c{c}\~oes diferencias parciais. Conforme \cite{pinho_2018}, podemos aplicar uma s\'erie de hip\'oteses oriundas das caracter\'isticas f\'isicas do efeito magneto-el\'astico, as quais visam simplificar e linearizar essas EDP's de forma que as mesmas possam receber um tratamento matem\'atico anal\'itico adequado, para em seguida se obter numericamente os valores dos campos eletromagn\'eticos e el\'asticos envolvidos no sistema. 
\begin{align}\label{eq.mag_ela_1}
\nabla\times\mathbf{{E}}&=i\,\omega\,\mu_0\mathbf{{H}}\\\nonumber\\\label{eq.mag_ela_2}
\nabla\times\mathbf{{H}}&=(\sigma-i\,\epsilon\,\omega)\,\mathbf{{E}}+\mathbf{{v}}\times\sigma\mu_0\mathbf{H}^0+\mathbf{j}\\\nonumber\\\label{eq.mag_ela_3}
-i\,\omega\rho\,\mathbf{{v}}&=\nabla\cdot{\tau} + \mathbf{{F}}\\\nonumber\\\label{eq.mag_ela_4}
{\tau}&=\lambda\,\nabla\cdot\mathbf{{u}}\cdot\,I + G\,(\nabla\,\mathbf{{u}}+\nabla\mathbf{{u}}^\top)\\\nonumber\\\label{eq.mag_ela_5}
\nabla\cdot\mathbf{{H}}&=0.
\end{align}
Estas equa\c{c}\~oes est\~ao no dom\'inio da frequ\^encia $\omega$, a depend\^encia do tempo \'e dada por $e^{-i\,\omega\,t}$ e 
\begin{itemize}
\item $\mathbf{{E}}$ \'e o campo el\'etrico,
\item $\mathbf{{B}}$ \'e o campo magn\'etico,
\item $\mathbf{{D}}$ \'e o campo de densidade de fluxo el\'etrico,
\item $\mathbf{{H}}$ \'e o campo magn\'etico auxiliar,
\item $\tau$ \'e o tensor de tens\~oes,
\item $\mathbf{{u}}$ \'e o deslocamento do meio,
\item $\mathbf{{v}}$ \'e a velocidade de deslocamento do meio,
\item $\mathbf{{F}}$ \'e uma for\c{c}a aplicada ao meio,
\item $\mathbf{H}^0$ \'e campo geomagn\'etico,
\item $i$ \'e um n\'umero complexo,
\item $\omega$ \'e a frequ\^encia temporal,
\item $\mu_0$ \'e a permeabilidade magn\'etica no v\'acuo,
\item $\sigma$ \'e a condutividade do meio,
\item $\epsilon$ \'e a permissividade el\'etrica do meio,
\item $\rho$ \'e a densidade do meio,
\item $\lambda$ e $G$ s\~ao par\^ametros de Lam\`e.
\end{itemize}
Vamos definir $\overline{\sigma}=(\sigma-i\,\epsilon\,\omega)$. No subsolo, por conta do regime quasi-estacion\'ario, $(\sigma>>\epsilon\,\omega)$  e  temos $\overline{\sigma}=\sigma$. No ar, a condutividade \'e zero e a permeabilidade el\'etrica \'e pr\'oxima a do v\'acuo $\epsilon_0$, assim temos $\overline{\sigma}=-i\,\epsilon_0\omega$.

No formato matricial, a equa\c{c}\~ao \ref{eq.mag_ela_1} pode ser escrita como
\begin{equation*}
\begin{pmatrix}
\frac{\partial\,E_3}{\partial\,y}-\frac{\partial\,E_2}{\partial\,z}\\
\frac{\partial\,E_1}{\partial\,z}-\frac{\partial\,E_3}{\partial\,x}\\
\frac{\partial\,E_2}{\partial\,x}-\frac{\partial\,E_1}{\partial\,y}
\end{pmatrix}
=
i\,\omega\,\mu_0\,
\begin{pmatrix}
H_1\\
H_2\\
H_3
\end{pmatrix}.
\end{equation*}
Aplicando as transformadas laterais de Fourier, dada em \ref{eq.trans_fourier_1}, temos
\begin{empheq}[left=\empheqlbrace]{align*}
i\,k_y\widehat{E}_3-\frac{\partial\,\widehat{E}_2}{\partial\,z}&=i\,\omega\,\mu_0\widehat{H}_1\\
\frac{\partial\,\widehat{E}_1}{\partial\,z}-i\,k_x\widehat{E}_3&=i\,\omega\,\mu_0\widehat{H}_2\\
i\,k_x\widehat{E}_2-i\,k_y\widehat{E}_1&=i\,\omega\,\mu_0\widehat{H}_3.
\end{empheq} 
Rotacionando o sistema de forma que a primeira coordenada esteja orientada no sentido do vetor de onda horizontal, usando o operador dado pela equa\c{c}\~ao \ref{eq.operador_rotacao}, e fazendo as simplifica\c{c}\~oes, temos
\begin{empheq}[left=\empheqlbrace]{align*}
-\frac{\partial\,\tilde{E}_2}{\partial\,z}&=i\,\omega\,\mu_0\tilde{H}_1\\
\frac{\partial\,\tilde{E}_1}{\partial\,z}&=i\,\omega\,\mu_0\tilde{H}_2+i\,k\tilde{E}_3\\
i\,\tilde{E}_2&=\frac{i\,\omega\,\mu_0}{k}\tilde{H}_3.
\end{empheq}

Observando que $\mathbf{v}=-i\,\omega\mathbf{u}$ depois de aplicada a transformada de Fourier no tempo, a equa\c{c}\~ao \ref{eq.mag_ela_2} pode ser escrita como
\begin{equation*}
\begin{pmatrix}
\frac{\partial\,H_3}{\partial\,y}-\frac{\partial\,H_2}{\partial\,z}\\
\frac{\partial\,H_1}{\partial\,z}-\frac{\partial\,H_3}{\partial\,x}\\
\frac{\partial\,H_2}{\partial\,x}-\frac{\partial\,H_1}{\partial\,y}
\end{pmatrix}
=
(\sigma-i\,\epsilon\,\omega)\,
\begin{pmatrix}
E_1\\
E_2\\
E_3
\end{pmatrix}
-i\,\omega\,\sigma\,\mu_0
\begin{pmatrix}
u_2H_3^0-u_3H_2^0\\
u_3H_1^0-u_1H_3^0\\
u_1H_2^0-u_2H_1^0
\end{pmatrix}
+
\begin{pmatrix}
j_1\\
j_2\\
j_3
\end{pmatrix}.
\end{equation*}
Aplicando as transformadas laterais de Fourier conforme a equa\c{c}\~ao \ref{eq.trans_fourier_1}, temos
\begin{empheq}[left=\empheqlbrace]{align*}
i\,k_y\widehat{H}_3-\frac{\partial\,\widehat{H}_2}{\partial\,z}&=(\sigma-i\,\epsilon\,\omega)\,\widehat{E}_1-i\,\omega\,\sigma\,\mu_0(u_2H_3^0-u_3H_2^0)+\widehat{j}_1\\
\frac{\partial\,\widehat{H}_1}{\partial\,z}-i\,k_x\widehat{H}_3&=(\sigma-i\,\epsilon\,\omega)\,\widehat{E}_2-i\,\omega\,\sigma\,\mu_0(u_3H_1^0-u_1H_3)+\widehat{j}_2\\
i\,k_x\widehat{H}_2-i\,k_y\widehat{H}_1&=(\sigma-i\,\epsilon\,\omega)\,\widehat{E}_3-i\,\omega\,\sigma\,\mu_0(u_1H_2^0-u_2H_1^0)+\widehat{j}_3.
\end{empheq}
Rotacionando o sistema usando o operador dado pela equa\c{c}\~ao \ref{eq.operador_rotacao}, e fazendo as simplifica\c{c}\~oes, temos
\begin{empheq}[left=\empheqlbrace]{align*}
-\frac{\partial\,\tilde{H}_2}{\partial\,z}&=(\sigma-i\,\epsilon\,\omega)\,\tilde{E}_1-i\,\omega\,\sigma\,\mu_0\,\tilde{H}_3^0\tilde{u}_2+i\,\omega\,\sigma\,\mu_0\,\tilde{H}_2^0\tilde{u}_3+\tilde{j}_1\\
\frac{\partial\,\tilde{H}_1}{\partial\,z}&=(\sigma-i\,\epsilon\,\omega)\,\tilde{E}_2+i\,k\,\tilde{H}_3-i\,\omega\,\sigma\,\mu_0\,\tilde{H}_1^0\tilde{u}_3+i\,\omega\,\sigma\,\mu_0\,\tilde{H}_3^0\tilde{u}_1+\tilde{j}_2\\
i\,k\,\tilde{H}_2&=(\sigma-i\,\epsilon\,\omega)\tilde{E}_3-i\,\omega\,\sigma\,\mu_0\,(\tilde{H}_2^0\tilde{u}_1-\tilde{H}_1^0\tilde{u}_2)+\tilde{j}_3.
\end{empheq}

A equa\c{c}\~ao \ref{eq.mag_ela_3} pode ser reescrita como

\begin{empheq}[left=\empheqlbrace]{align*}
-\omega^2\rho\,u_1&=\frac{\partial \tau_{11}}{\partial\,x}+\frac{\partial \tau_{12}}{\partial\,y}+\frac{\partial \tau_{13}}{\partial\,z}+F_1\\
-\omega^2\rho\,u_2&=\frac{\partial \tau_{21}}{\partial\,x}+\frac{\partial \tau_{22}}{\partial\,y}+\frac{\partial \tau_{23}}{\partial\,z}+F_2\\
-\omega^2\rho\,u_3&=\frac{\partial \tau_{31}}{\partial\,x}+\frac{\partial \tau_{32}}{\partial\,y}+\frac{\partial \tau_{33}}{\partial\,z}+F_3
\end{empheq}
Aplicando as transformadas laterais de Fourier, temos
\begin{empheq}[left=\empheqlbrace]{align*}
-\omega^2\rho\,\widehat{u}_1&=i\,k_x\widehat{\tau}_{11}+i\,k_y\widehat{\tau}_{12}+\frac{\partial\widehat{\tau}_{13}}{\partial\,z}+\widehat{F}_1\\
-\omega^2\rho\,\widehat{u}_2&=i\,k_x\widehat{\tau}_{21}+i\,k_y\widehat{\tau}_{22}+\frac{\partial\widehat{\tau}_{23}}{\partial\,z}+\widehat{F}_2\\
-\omega^2\rho\,\widehat{u}_3&=i\,k_x\widehat{\tau}_{31}+i\,k_y\widehat{\tau}_{32}+\frac{\partial\widehat{\tau}_{33}}{\partial\,z}+\widehat{F}_3.
\end{empheq}
Aplicando a rota\c{c}\~ao e simplificando as equa\c{c}\~oes, temos
\begin{empheq}[left=\empheqlbrace]{align*}
-\omega^2\rho\,\tilde{u}_1&=i\,k\tilde{\tau}_{11}+\frac{\partial\tilde{\tau}_{13}}{\partial\,z}+\tilde{F}_1\\
-\omega^2\rho\,\tilde{u}_2&=i\,k\tilde{\tau}_{12}+\frac{\partial\tilde{\tau}_{23}}{\partial\,z}+\tilde{F}_2\\
-\omega^2\rho\,\tilde{u}_3&=i\,k\tilde{\tau}_{13}+\frac{\partial\tilde{\tau}_{33}}{\partial\,z}+\tilde{F}_3.
\end{empheq}

A equa\c{c}\~ao \ref{eq.mag_ela_4} pode ser reescrita como
\begin{equation*}
\begin{pmatrix}
\tau_{11}&\tau_{12}&\tau_{13}\\
\tau_{21}&\tau_{22}&\tau_{23}\\
\tau_{31}&\tau_{32}&\tau_{33}
\end{pmatrix}
=
\lambda\,\left(\frac{\partial\,u_1}{\partial\,x}+\frac{\partial\,u_2}{\partial\,y}+\frac{\partial\,u_3}{\partial\,z}\right)\,I+G\,
\begin{pmatrix}
2\,\frac{\partial\,u_1}{\partial\,x}&\frac{\partial\,u_1}{\partial\,y}+\frac{\partial\,u_2}{\partial\,x}&\frac{\partial\,u_1}{\partial\,z}+\frac{\partial\,u_3}{\partial\,x}\\
\frac{\partial\,u_2}{\partial\,x}+\frac{\partial\,u_1}{\partial\,y}&2\,\frac{\partial\,u_2}{\partial\,y}&\frac{\partial\,u_2}{\partial\,z}+\frac{\partial\,u_3}{\partial\,y}\\
\frac{\partial\,u_3}{\partial\,x}+\frac{\partial\,u_1}{\partial\,z}&\frac{\partial\,u_3}{\partial\,y}+\frac{\partial\,u_2}{\partial\,z}&2\,\frac{\partial\,u_3}{\partial\,z}
\end{pmatrix},
\end{equation*}
onde $I$ \'e uma matriz identidade.
Dada a simetria do tensor $\tau_{ij}$ temos seis equa\c{c}\~oes
\begin{empheq}[left=\empheqlbrace]{align*}
\tau_{11}&=\lambda\,\left(\frac{\partial\,u_1}{\partial\,x}+\frac{\partial\,u_2}{\partial\,y}+\frac{\partial\,u_3}{\partial\,z}\right)+2\,G\,\frac{\partial\,u_1}{\partial\,x}\\
\tau_{12}&=G\,\left(\frac{\partial\,u_1}{\partial\,y}+\frac{\partial\,u_2}{\partial\,x}\right)\\
\tau_{13}&=G\,\left(\frac{\partial\,u_1}{\partial\,z}+\frac{\partial\,u_3}{\partial\,x}\right)\\
\tau_{22}&=\lambda\,\left(\frac{\partial\,u_1}{\partial\,x}+\frac{\partial\,u_2}{\partial\,y}+\frac{\partial\,u_3}{\partial\,z}\right)+2\,G\,\frac{\partial\,u_2}{\partial\,y}\\
\tau_{23}&=G\,\left(\frac{\partial\,u_2}{\partial\,z}+\frac{\partial\,u_3}{\partial\,y}\right)\\
\tau_{33}&=\lambda\,\left(\frac{\partial\,u_1}{\partial\,x}+\frac{\partial\,u_2}{\partial\,y}+\frac{\partial\,u_3}{\partial\,z}\right)+2\,G\,\frac{\partial\,u_3}{\partial\,z}.
\end{empheq}
Aplicando as transformadas laterais de Fourier no sistema acima, temos
\begin{empheq}[left=\empheqlbrace]{align*}
\widehat{\tau}_{11}&=\lambda\,\left(-i\,k_x\widehat{u}_1-i\,k_y\widehat{u}_2+\frac{\partial\,\widehat{u}_3}{\partial\,z}\right)-2\,i\,G\,k_x\widehat{u}_1\\
\widehat{\tau}_{12}&=G\,\left(-i\,k_y\widehat{u}_1-i\,k_x\widehat{u}_2\right)\\
\widehat{\tau}_{13}&=G\,\left(\frac{\partial\,\widehat{u}_1}{\partial\,z}-i\,k_x\widehat{u}_3\right)\\
\widehat{\tau}_{22}&=\lambda\,\left(-i\,k_x\widehat{u}_1-i\,k_y\widehat{u}_2+\frac{\partial\,\widehat{u}_3}{\partial\,z}\right)-2\,i\,G\,k_y\widehat{u}_2\\
\widehat{\tau}_{23}&=G\,\left(\frac{\partial\,\widehat{u}_2}{\partial\,z}-i\,k_y\widehat{u}_3\right)\\
\widehat{\tau}_{33}&=\lambda\,\left(-i\,k_x\widehat{u}_1-i\,k_y\widehat{u}_2+\frac{\partial\,\widehat{u}_3}{\partial\,z}\right)+2\,G\,\frac{\partial\,\widehat{u}_3}{\partial\,z}.
\end{empheq}
E aplicando a rota\c{c}\~ao $\Omega$ na forma dada pela equa\c{c}\~ao \ref{eq.rotacao_tensor}, fazendo as simplifica\c{c}\~oes, finalmente a lei de Hooke se torna
\begin{empheq}[left=\empheqlbrace]{align*}
\tilde{\tau}_{11}&=i\,k(\lambda+2\,G)\,\tilde{u}_1+\lambda\,\frac{\partial\,\tilde{u}_3}{\partial\,z}\\
\tilde{\tau}_{12}&=i\,G\,k\,\tilde{u}_2\\
\tilde{\tau}_{13}&=G\,\frac{\partial\,\tilde{u}_1}{\partial\,z}+i\,G\,k\,\tilde{u}_3\\
\tilde{\tau}_{22}&=\lambda\,\frac{\partial\,\tilde{u}_3}{\partial\,z}+i\,\lambda\,k\,\tilde{u}_1\\
\tilde{\tau}_{23}&=G\,\frac{\partial\,\tilde{u}_2}{\partial\,z}\\
\tilde{\tau}_{33}&=(\lambda+2\,G)\,\frac{\partial\,\tilde{u}_3}{\partial\,z}+i\,\lambda\,k\,\tilde{u}_1.
\end{empheq}
Vimos que as equa\c{c}\~oes \ref{eq.mag_ela_1}, \ref{eq.mag_ela_2}, \ref{eq.mag_ela_3} e \ref{eq.mag_ela_4} escritas no espa\c{c}o horizontal de Fourier e rotacionadas nos fornecem quinze equa\c{c}\~oes. Isolando as vari\'aveis $\tilde{H}_3,\tilde{E}_3,\tilde{\tau}_{11},\tilde{\tau}_{22}$ e $\tilde{\tau}_{12}$, podemos substituir algumas equa\c{c}\~oes em outras e reduzir a quantidade para dez. Fazendo ainda as substitui\c{c}\~oes da vari\'avel $\dot{\tilde{\mathbf{u}}}=-i\,\omega\,\tilde{\mathbf{u}}$, da vagarosidade horizontal conforme a equa\c{c}\~ao \ref{eq.numero_onda_vagarozidade_horizontal} e isolando as derivadas parciais em rela\c{c}\~ao \`a profundidade, temos o seguinte sistema 
%\begin{empheq}[left=\empheqlbrace]{align}
\begin{align}
\frac{\partial\,\tilde{E}_1}{\partial\,z}&=\left(i\,\omega\,\mu_0+\frac{i^2\gamma^2\omega^2}{\sigma-i\,\epsilon\,\omega}\right)\,\tilde{H}_2-\frac{i\,\gamma\,\omega\,\sigma\,\mu_0\tilde{H}_2^0}{\sigma-i\,\epsilon\,\omega}\dot{\tilde{u}}_1+\frac{i\,\gamma\,\omega\,\sigma\,\mu_0\tilde{H}_1^0}{\sigma-i\,\epsilon\,\omega}\dot{\tilde{u}}_2+\frac{i\,\gamma\,\omega}{\sigma-i\,\epsilon\,\omega}\,\tilde{j}_3\\\nonumber\\\label{eq.edo_2}
\frac{\partial\,\tilde{E}_2}{\partial\,z}&=-i\,\omega\,\mu_0\,\tilde{H}_1\\\nonumber\\\label{eq.edo_3}
-\frac{\partial\,\tilde{H}_1}{\partial\,z}&=-\left(\sigma-i\,\epsilon\,\omega+\frac{i^2\gamma^2\omega^2}{i\,\omega\,\mu_0}\right)\,\tilde{E}_2+\sigma\,\mu_0\tilde{H}_3^0\dot{\tilde{u}}_1-\sigma\,\mu_0\tilde{H}_1^0\dot{\tilde{u}}_3-\tilde{j}_2\\\nonumber\\
\frac{\partial\,\tilde{H}_2}{\partial\,z}&=-\left(\sigma-i\,\epsilon\,\omega\right)\,\tilde{E}_1-\sigma\,\mu_0\tilde{H}_3^0\dot{\tilde{u}}_2+\sigma\,\mu_0\tilde{H}_2^0\dot{\tilde{u}}_3-\tilde{j}_1\\\nonumber\\\label{eq.edo_5}
\frac{\partial\,\dot{\tilde{u}}_1}{\partial\,z}&=-i\,\gamma\,\omega\,\dot{\tilde{u}}_3-\frac{i\,\omega}{G}\,\tilde{\tau}_{13}\\\nonumber\\
\frac{\partial\,\dot{\tilde{u}}_2}{\partial\,z}&=-\frac{i\,\omega}{G}\,\tilde{\tau}_{23}\\\nonumber\\\label{eq.edo_7}
\frac{\partial\,\dot{\tilde{u}}_3}{\partial\,z}&=-\frac{i\,\omega}{\lambda+2\,G}\,\tilde{\tau}_{33}-\frac{i\,\omega\,\gamma\,\lambda}{\lambda+2\,G}\,\dot{\tilde{u}}_1\\\nonumber\\
\frac{\partial\,\tilde{\tau}_{13}}{\partial\,z}&=
\left[\frac{\omega\,\rho}{i}-\frac{i^2\gamma^2\omega}{i}\cdot\frac{\lambda^2-(\lambda+2\,G)^2}{\lambda+2\,G}\right]\,\dot{\tilde{u}}_1-\frac{i\,\omega\,\gamma\,\lambda}{\lambda+2\,G}\,\tilde{\tau}_{33}-\tilde{F}_1\\\nonumber\\
\frac{\partial\,\tilde{\tau}_{23}}{\partial\,z}&=\left[\frac{\omega\,\rho}{i}+\frac{i^2\gamma^2\omega\,G}{i}\right]\,\dot{\tilde{u}}_2-\tilde{F}_2\\\nonumber\\\label{eq.edo_10}
\frac{\partial\,\tilde{\tau}_{33}}{\partial\,z}&=\frac{\omega\,\rho}{i}\,\dot{\tilde{u}}_3-i\,\omega\,\gamma\,\tilde{\tau}_{13}-\tilde{F}_3.
\end{align}
%\end{empheq}

%SISTEMA COMO TODO NO FORMATO MATRICIAL

%
%\begin{landscape}
%\begin{equation*}
%\frac{\partial}{\partial\,z}
%\begin{pmatrix}
%\tilde{E}_1\\\\
%\tilde{E}_2\\\\
%\tilde{H}_1\\\\
%\tilde{H}_2\\\\
%\dot{\tilde{u}}_1\\\\
%\dot{\tilde{u}}_2\\\\
%\dot{\tilde{u}}_3\\\\
%\tilde{\tau}_{13}\\\\
%\tilde{\tau}_{23}\\\\
%\tilde{\tau}_{33}
%\end{pmatrix}
%=
%\begin{pmatrix}
%0&0&0&i\,\omega\,\mu_0-\frac{k^2}{\sigma-i\,\epsilon\,\omega}&-\frac{k\,\sigma\,\mu_0\tilde{H}_2^0}{i\,(\sigma-i\,\epsilon\,\omega)}&\frac{k\,\sigma\,\mu_0\tilde{H}_1^0}{i\,(\sigma-i\,\epsilon\,\omega)}&0&0&0&0\\\\
%0&0&-i\,\omega\,\mu_0&0&0&0&0&0&0&0\\\\
%0&\sigma-i\,\epsilon\,\omega+\frac{i\,k^2}{\omega\,\mu_0}&0&0&-\sigma\,\mu_0\tilde{H}_3^0&0&\sigma\,\mu_0\tilde{H}_1^0&0&0&0\\\\
%-(\sigma-i\,\epsilon\,\omega)&0&0&0&0&-\sigma\,\mu_0\tilde{H}_3^0&\sigma\,\mu_0\tilde{H}_2^0&0&0&0\\\\
%0&0&0&0&0&0&i\,k&0&0&-\frac{i\,\omega}{G}\\\\
%0&0&0&0&0&0&0&0&-\frac{i\,\omega}{G}&0\\\\
%0&0&0&0&\frac{i\,\lambda\,k}{\lambda+2\,G}&0&0&0&0&-\frac{i\,\omega}{\lambda+2\,G}\\\\
%0&0&0&0&\frac{\omega\rho}{i}-\frac{k^2}{i\,\omega}\,\frac{(\lambda+2\,G)^2-\lambda^2}{\lambda+2\,G}&0&0&0&0&\frac{i\,\lambda\,k}{\lambda+2\,G}\\\\
%0&0&0&0&0&\omega\rho-\frac{G\,k^2}{i\,\omega}&0&0&0&0\\\\
%0&0&0&0&0&0&\frac{\omega\rho}{i}&i\,k&0&0
%\end{pmatrix}
%\begin{pmatrix}
%\tilde{E}_1\\\\
%\tilde{E}_2\\\\
%\tilde{H}_1\\\\
%\tilde{H}_2\\\\
%\dot{\tilde{u}}_1\\\\
%\dot{\tilde{u}}_2\\\\
%\dot{\tilde{u}}_3\\\\
%\tilde{\tau}_{13}\\\\
%\tilde{\tau}_{23}\\\\
%\tilde{\tau}_{33}
%\end{pmatrix}
%+
%\begin{pmatrix}
%-\frac{i\,k^2}{\sigma-i\epsilon\omega}\,\tilde{\mathbf{j}}_3\\\\
%0\\\\
%\tilde{\mathbf{j}}_2\\\\
%-\tilde{\mathbf{j}}_1\\\\
%0\\\\
%0\\\\
%0\\\\
%-\tilde{F}_1\\\\
%-\tilde{F}_2\\\\
%-\tilde{F}_3
%\end{pmatrix}
%\end{equation*}
%\end{landscape}

% SISTEMA INICIAL SEPARADO EM DOIS E EM TERMOS DA VAGAROSIDADE HORIZONTAL

%Podemos escrever os sistemas em termos da vagarozidade horizontal, de acordo com a equacao \ref{eq.numero_onda_vagarozidade_horizontal}.
%\begin{landscape}
%\begin{equation*}
%\frac{\partial}{\partial\,z}
%\begin{pmatrix}
%\tilde{E}_1\\\\
%\tilde{E}_2\\\\
%\tilde{H}_1\\\\
%\tilde{H}_2
%\end{pmatrix}
%=-i\,\omega\,
%\begin{pmatrix}
%0&0&0&-\mu_0+\frac{\gamma\,\omega}{i(\sigma-i\,\epsilon\,\omega)}\\\\
%0&0&\mu_0&0\\\\
%0&-\frac{\sigma-i\,\epsilon\,\omega}{i\,\omega}-\frac{\gamma^2}{\mu_0}&0&0\\\\
%\frac{\sigma-i\,\epsilon\,\omega}{i\,\omega}&0&0&0
%\end{pmatrix}
%\begin{pmatrix}
%\tilde{E}_1\\\\
%\tilde{E}_2\\\\
%\tilde{H}_1\\\\
%\tilde{H}_2
%\end{pmatrix}
%+
%\begin{pmatrix}
%\frac{\gamma\,\omega\,\sigma\,\mu_0\tilde{H}_1^0}{i\,(\sigma-i\,\epsilon\,\omega)}\dot{\tilde{u}}_2-\frac{\gamma\,\omega\,\sigma\,\mu_0\tilde{H}_2^0}{i\,(\sigma-i\,\epsilon\,\omega)}\dot{\tilde{u}}_1-\frac{i\,(\gamma\,\omega)^2}{\sigma-i\epsilon\omega}\,\tilde{j}_3\\\\
%0\\\\
%\sigma\,\mu_0\tilde{H}_1^0\dot{\tilde{u}}_3-\sigma\,\mu_0\tilde{H}_3^0\dot{\tilde{u}}_1 
%+\tilde{j}_2\\\\
%\sigma\,\mu_0\tilde{H}_2^0\dot{\tilde{u}}_3-\sigma\,\mu_0\tilde{H}_3^0\dot{\tilde{u}}_2-\tilde{j}_1
%\end{pmatrix}
%\end{equation*}\\
%\begin{equation*}
%\frac{\partial}{\partial\,z}
%\begin{pmatrix}
%\dot{\tilde{u}}_1\\\\
%\dot{\tilde{u}}_2\\\\
%\tilde{\tau}_{33}\\\\
%\dot{\tilde{u}}_3\\\\
%\tilde{\tau}_{13}\\\\
%\tilde{\tau}_{23}
%\end{pmatrix}
%=-i\,\omega\,
%\begin{pmatrix}
%0&0&0&-\gamma&\frac{1}{G}&0&\\\\
%0&0&0&0&0&\frac{1}{G}\\\\
%0&0&0&\rho&-\gamma&0\\\\
%-\frac{\lambda\,\gamma}{\lambda+2\,G}&0&\frac{1}{\lambda+2\,G}&0&0&0\\\\
%\rho-\gamma^2\frac{(\lambda+2\,G)^2-\lambda^2}{\lambda+2\,G}&0&-\frac{\lambda\,\gamma}{\lambda+2\,G}&0&0&0\\\\
%0&-\frac{\rho}{i}-G\,\gamma^2&0&0&0&0\\\\
%\end{pmatrix}
%\begin{pmatrix}
%\dot{\tilde{u}}_1\\\\
%\dot{\tilde{u}}_2\\\\
%\tilde{\tau}_{33}\\\\
%\dot{\tilde{u}}_3\\\\
%\tilde{\tau}_{13}\\\\
%\tilde{\tau}_{23}\\\\
%\end{pmatrix}
%+
%\begin{pmatrix}
%0\\\\
%0\\\\
%-\tilde{F}_3\\\\
%0\\\\
%-\tilde{F}_1\\\\
%-\tilde{F}_2
%\end{pmatrix}
%\end{equation*}
%\end{landscape}

O sistema acima pode ser colocado num formato matricial composto por submatrizes nulas e submatrizes sim\'etricas, e para isso precisamos agrupar essa dez equa\c{c}\~oes em quatro equa\c{c}\~oes matriciais.

INCLUDE HERE THE NEW PARTITION FOR  ODE'S AND CREATE THE TWO NEW SYSTEMS. MAYBE IT'S INTERESTING START ANOTHER CHAPTER OR SECTION. THE PREVIOUS INFORMATION MUST BE THE COMMON ONE FOR BOTH PROPAGATION APPROACH: ONE DIMENSION AND GENERAL CASES.  

\begin{align}\label{eq.matricial_1}
\frac{\partial}{\partial\,z}
\begin{pmatrix}
\tilde{E}_1\\\\
\tilde{H}_2
\end{pmatrix}
&=-i\,\omega\,
\begin{pmatrix}
0&-\mu_0-\frac{i\,\gamma^2\omega}{\sigma-i\,\epsilon\,\omega}\\\\
\frac{\sigma-i\,\epsilon\,\omega}{i\,\omega}&0
\end{pmatrix}
\begin{pmatrix}
\tilde{E}_1\\\\
\tilde{H}_2
\end{pmatrix}
+
\begin{pmatrix}
\frac{i\,\gamma\,\omega\,\sigma\,\mu_0\tilde{H}_1^0}{\sigma-i\,\epsilon\,\omega}\dot{\tilde{u}}_2-\frac{i\,\gamma\,\omega\,\sigma\,\mu_0\tilde{H}_2^0}{\sigma-i\,\epsilon\,\omega}\dot{\tilde{u}}_1+\frac{i\,\gamma\,\omega}{\sigma-i\epsilon\omega}\,\tilde{j}_3\\\\
\sigma\,\mu_0\tilde{H}_2^0\dot{\tilde{u}}_3-\sigma\,\mu_0\tilde{H}_3^0\dot{\tilde{u}}_2-\tilde{j}_1
\end{pmatrix}\\\nonumber\\\label{eq.matricial_2}
\frac{\partial}{\partial\,z}
\begin{pmatrix}
\tilde{E}_2\\\\
-\tilde{H}_1
\end{pmatrix}
&=-i\,\omega\,
\begin{pmatrix}
0&-\mu_0\\\\
\frac{\sigma-i\,\epsilon\,\omega}{i\,\omega}+\frac{i\,\omega\,\gamma^2}{i\,\omega\,\mu_0}&0
\end{pmatrix}
\begin{pmatrix}
\tilde{E}_2\\\\
-\tilde{H}_1
\end{pmatrix}
+
\begin{pmatrix}
0\\\\
-\sigma\,\mu_0\tilde{H}_1^0\dot{\tilde{u}}_3+\sigma\,\mu_0\tilde{H}_3^0\dot{\tilde{u}}_1 
-\tilde{j}_2
\end{pmatrix}\\\nonumber\\\label{eq.matricial_3}
\frac{\partial}{\partial\,z}
\begin{pmatrix}
\dot{\tilde{u}}_3\\\\
\tilde{\tau}_{13}\\\\
\tilde{\tau}_{33}\\\\
\dot{\tilde{u}}_1\\\\
\end{pmatrix}
&=-i\,\omega\,
\begin{pmatrix}
0&0&\frac{1}{\lambda+2\,G}&\frac{\lambda\,\gamma}{\lambda+2\,G}\\\\
0&0&\frac{\lambda\,\gamma}{\lambda+2\,G}&\rho+\gamma^2\frac{\lambda^2-(\lambda+2\,G)^2}{\lambda+2\,G}\\\\
\rho&\gamma&0&0\\\\
\gamma&\frac{1}{G}&0&0\\\\
\end{pmatrix}
\begin{pmatrix}
\dot{\tilde{u}}_3\\\\
\tilde{\tau}_{13}\\\\
\tilde{\tau}_{33}\\\\
\dot{\tilde{u}}_1\\\\
\end{pmatrix}
+
\begin{pmatrix}
0\\\\
-\tilde{F}_1\\\\
-\tilde{F}_3\\\\
0\\\\
\end{pmatrix}\\\nonumber\\\label{eq.matricial_4}
\frac{\partial}{\partial\,z}
\begin{pmatrix}
\dot{\tilde{u}}_2\\\\
\tilde{\tau}_{23}
\end{pmatrix}
&=-i\,\omega\,
\begin{pmatrix}
0&\frac{1}{G}\\\\
\rho-G\,\gamma^2&0\\\\
\end{pmatrix}
\begin{pmatrix}
\dot{\tilde{u}}_2\\\\
\tilde{\tau}_{23}\\\\
\end{pmatrix}
+
\begin{pmatrix}
0\\\\
-\tilde{F}_2
\end{pmatrix}
\end{align}
Uma vez que as dez vari\'aveis das equa\c{c}\~oes matricias acima tenham sido determinadas, podemos determinar tamb\'em as cinco vari\'aveis restantes usando
\begin{empheq}[left=\empheqlbrace]{align*}
\tilde{H}_3&=\frac{\gamma}{\mu_0}\,\tilde{E}_2\\
\tilde{E}_3&=\frac{i\,\omega\,\gamma}{\overline{\sigma}}\,\tilde{H}_2-\frac{\sigma\,\mu_0\tilde{H}_2^0}{\overline{\sigma}}\,\dot{\tilde{u}}_1+\frac{\sigma\,\mu_0\tilde{H}_1^0}{\overline{\sigma}}\,\dot{\tilde{u}}_2-\frac{1}{\overline{\sigma}}\,\tilde{j}_3\\
\tilde{\tau}_{11}&=\gamma\,\frac{\lambda^2-(\lambda+2\,G)^2}{\lambda+2\,G}\dot{\tilde{u}}_1+\frac{\lambda}{\lambda+2\,G}\,\tilde{\tau}_{33}\\
\tilde{\tau}_{22}&=\gamma\,\lambda\,\frac{\lambda-(\lambda+2\,G)}{\lambda+2\,G}\dot{\tilde{u}}_1+\frac{\lambda}{\lambda+2\,G}\,\tilde{\tau}_{33}\\
\tilde{\tau}_{12}&=-\gamma\,G\,\dot{\tilde{u}}_2.
\end{empheq}

