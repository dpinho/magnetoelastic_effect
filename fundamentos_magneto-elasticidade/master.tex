\documentclass[12pt,a4paper,oneside]{abntex2}
\usepackage[utf8]{inputenc}
\usepackage{amsmath}
\usepackage{esint}
\usepackage{amsfonts}
\usepackage{amssymb}
\usepackage{graphicx}
\usepackage{subfig}
\graphicspath{{fig/}}
\providecommand{\norm}[1]{\lVert#1\rVert}
\usepackage{empheq}

%\usepackage[alf ,abnt-etal-cite=2 , abnt-year-extra-label=yes , abnt-etal-list=0, abnt-etal-text=it]{abntex2cite}

%capa
\autor{DAVID DA COSTA DE PINHO}% introduz nome do autor
\titulo{FUNDAMENTOS DE MAGNETO-ELASTICIDADE}
\data{2018}
\local{MACA\'E}

%anverso da folha de rosto
\preambulo{Monografia apresentada ao Centro de Ci\^encia e Tecnologia da Universidade Estadual do Norte Fluminense, como parte das exig\^encias para aprova\c{c}\~ao no Exame de Qualifica\c{c}\~ao.}
\orientador{Viatcheslav Ivanovich Priimenko, Ph.D}
\tipotrabalho{monografia}

\begin{document}

\imprimircapa

\imprimirfolhaderosto%anverso

%{\ABNTEXchapterfont\Large\textsc{\imprimirautor}} - para aumentar a fonte e colocar em caixa alta

%importante dar enter para pular linha entre os tipos de textos

\begin{folhadeaprovacao}

\begin{center}
{\ABNTEXchapterfont\Large\bfseries\imprimirtitulo}\\\vspace{1cm}
{\ABNTEXchapterfont\Large\textsc{\imprimirautor}}

\end{center}

\vspace{1cm}
\hspace{.45\textwidth} \begin{minipage}{.5\textwidth}
\imprimirpreambulo
\end{minipage}

\vspace{1cm}
Trabalho aprovado em $\,7\,$ de Agosto de \imprimirdata.\\

\vspace{1cm}
Comiss\~ao Examinadora:\\
\assinatura{Prof. Marcia Miranda Azeredo, Ds.c - UENF}
\assinatura{Prof. André Duarte Bueno, Ds.c - UENF}
\assinatura{Prof. Fernando Diogo de Siqueira, Ds.c - UENF}
\assinatura{Prof. \imprimirorientador - UENF}

\begin{center}
\vfill
{\large\imprimirlocal}
\par
{\large\imprimirdata}
\end{center}

\end{folhadeaprovacao}

\begin{center}
{\ABNTEXchapterfont\Large\imprimirtitulo}\\\vspace{1cm}

\end{center}

\begin{resumo}
O objetivo principal desta obra \'e apresentar os fundamentos das teorias f\'isicas e matem\'atica relacionados ao efeito magneto-el\'astico, bem como um recondicionamento do modelo usado para descrever tal efeito. Inicialmente, apresentamos uma revis\~ao dos principais fen\^omenos pesquisados atualmente, envolvendo propaga\c{c}\~ao de ondas el\'asticas e eletromagn\'eticas. A mec\^anica do cont\'inuo e o eletromagnetismo s\~ao as duas teorias f\'isicas que descrevem a magneto-elasticidade, e introduzimos os principais conceitos dessas \'areas necess\'arios aos propositos desta monografia. Estudamos como ocorre o acoplamento entre ondas el\'asticas e eletromagn\'eticas bem como sua propaga\c{c}\~ao na subsuperf\'icie terrestre, atrav\'es de camadas estratigr\'aficas. Finalizamos com uma reescrita do modelo desse acoplamento de forma que possamos desenvolv\^e-lo analiticamente e produzir uma solu\c{c}\~ao num\'erica em trabalhos posteriores.

\vspace{\onelineskip} 
\noindent 
\textbf{Palavras-chave}: Magneto-Elasticidade, Mec\^anica do Cont\'inuo, Equa\c{c}\~oes de Maxwell.
\end{resumo}

\begin{center}
{\ABNTEXchapterfont\Large FOUNDATIONS OF MAGNETO-ELASTICITY}\\\vspace{1cm}
\end{center}

\begin{resumo}[Abstract]
\begin{otherlanguage*}{english}
The main purpose of this paper is present the foundations of physical and mathematical theories about magneto-elastic effect, and a reconditioning model to describe such effect. Firstly, we present a review of some phenomena found in current surveys, involving the propagations of elastic and eletromagnetics waves. Continuous mechanics and eletromagnetism describe magneto-elasticity, and we introduce a lots of required concepts from these areas to achieve the monography's goals. We also study how occurs the coupled propagation of elastic and eletromagnetics waves in Earth's subsurface, through stratigraphics layers. We finished rewriting the coupling's model in such manner that allow us to develope it analiticaly and to build a numerical solver in further works.

\vspace{\onelineskip} 
\noindent 
\textbf{Keywords}: Magneto-Elasticity, Continuous Mechanics, Maxwell's Equations.

\end{otherlanguage*}
\end{resumo}

\listoffigures

\newpage

\tableofcontents

\textual

\chapter{Introdu\c{c}\~ao e Objetivos}
Esta monografia tem como finalidade principal o detalhamento da teoria que envolve o acoplamento de ondas eletromagn\'eticas e ondas el\'asticas que se propagam em meios estratificados, e homog\^eneo por camada, no subsolo terrestre. O desenvolvimento dessa teoria segue o modelo apresentado por \cite{erigen_1963} que trata da propaga\c{c}\~ao de ondas el\'asticas num campo eletromagn\'etico (geomagn\'etico), onde essa propaga\c{c}\~ao gera pequenas altera\c{c}\~oes geomagn\'eticas que se propagam, n\~ao com a velocidade da luz, mas ``acompanhando'' a onda el\'astica mantendo a velocidade desta \'ultima. 

A teoria é essencialmente uma combinação de elasticidade infinitesimal e teoria eletromagnética linearizada, e para tornar o texto o mais auto-suficiente poss\'ivel, serão apresentados nos capítulos \ref{sec.fund_eletr} e \ref{sec.fund_elast} os principais conceitos e definições acerca das teorias básicas sobre eletromagnetismo e elasticidade importantes para o efeito magneto-el\'astico.

Esta monografia faz parte de um conjunto de pesquisas que objetivam desenvolver um novo modelo matemático-computacional para descrever os fenômenos que envolvem a propagação simultânea de ondas eletromagnéticas e elásticas em subsuperfície, de modo que tal levantamento possa ser usado para aprimorar as técnicas de exploração de petróleo ou outro bem mineral.
\input{revisao_bibliografica_1}
\chapter{Fundamentos de eletromagnetismo}\label{sec.fund_eletr}

\section{Introdução}

\section{Fatos experimentais}

\subsection{Lei de Gauss para os fluxos elétrico e magnético}
De acordo com \cite{jackson_classical_1999} e \cite{sommerfeld_52} , os conceitos, definições e resultados em eletromagnetismo clássico partem das experiências de Cavendish e Coulomb no final do Séc. $XVIII$. A partir desses experimentos foi estabelecida a Lei de Coulomb
\begin{equation}\label{eq.forc_elet}
\textbf{F}=k\,\frac{q_1\,q_2}{||\textbf{x}_1-\textbf{x}_2||^2}\frac{\textbf{x}_1-\textbf{x}_2}{||\textbf{x}_1-\textbf{x}_2||},
\end{equation}
onde $q_i$ são as cargas elétricas (campos escalares) presentes nos pontos $\textbf{x}_i$, respectivamente, $k$ (campo escalar) é uma constante de proporcionalidade cujo valor depende do sistema de unidades de medida adotado, $||\textbf{x}_1-\textbf{x}_2||^2$ é a distância euclidiana entre as cargas e $\textbf{F}$ é a força elétrica exercida pela carga $q_1$ sobre a carga $q_2$. As notações em negrito representam campos vetorias pertencentes ao espaço $\mathbb{R}^3$, e o vetor normal que fornece a direção de interação entre as cargas é dado por $(\textbf{x}_1-\textbf{x}_2)/||\textbf{x}_1-\textbf{x}_2||$.

O campo elétrico $\textbf{E}$ é definido como sendo a força elétrica por unidade de carga em um determinado ponto que contém a carga de prova $q_2$, portanto é uma função vetorial que depende da posição da carga de prova em relação à carga fonte $q_1$, ou seja,
\begin{equation}\label{eq.camp_elet}
\textbf{E}=\lim_{q_2\to 0}\frac{\textbf{F}}{q_2}.
\end{equation}
A carga de prova foi tomada infinitesimalmente pequena para que o campo gerado por ela não perturbe a carga fonte. Experimentalmente, tanto a direção da força como a razão entre a força e a quantidade de carga vão se tornando constantes à medida que a quantidade de carga se torna cada vez menor, definindo a magnitude e a direção do campo elétrico. No SI, a unidade de medida de carga é o \textit{coulomb} $(C)$, o campo elétrico é o \textit{newton/coulomb} $(N/C)$ ou o \textit{volt/metro} $(V/m)$, e a constante $k=(4\pi\,\epsilon_0)^{-1}$ onde $\epsilon_0\simeq8.854\times10^{-12}$ é a permissividade elétrica no vácuo medida em \textit{farad/m} $(F/m)$.

Substituindo a equação \ref{eq.camp_elet} em \ref{eq.forc_elet} temos que o campo elétrico agindo num ponto $\textbf{x}$ qualquer devido a uma carga $q_1$ no ponto $\textbf{x}_1$ é
\begin{equation}\label{eq.campo_eletrico}
\textbf{E}=k\,\frac{q_1}{||\textbf{x}_1-\textbf{x}||^2}\frac{\textbf{x}_1-\textbf{x}}{||\textbf{x}_1-\textbf{x}||},
\end{equation}
como podemos observar na figura \ref{fig.camp_eletr} simulando um sistema de coordenadas qualquer.
\begin{figure}[!htb]
\centering
\includegraphics[scale=1.5]{camp_elet}
\caption{\textit{Exemplificação da interação entre cargas elétricas devido à geração, em função de $q_1$, de um campo elétrico. A força elétrica $\textbf{F}$ atuando numa carga qualquer $q$ tem mesma direção do campo elétrico $\textbf{E}$, com mesmo sentido ou sentido oposto conforme a carga $q$ é positiva ou negativa, respectivamente.}}
\label{fig.camp_eletr}
\end{figure}

%\begin{figure}[!htb]
%\centering
%\subfloat{\includegraphics[scale=1]{camp_elet_3D}}
%\subfloat{\includegraphics[scale=1]{camp_elet_3D2}}
%\caption{}
%\label{fig.mossul}
%\end{figure}

Num sistema com mais de uma carga fonte produzindo campos elétricos, foi observado experimentalmente que o campo elétrico total atuando num ponto $\textbf{x}$ é simplesmente o somatório dos campos produzidos por cada carga, o que ficou conhecido como a \textit{Superposição Linear} e pode ser expressada na forma
\begin{equation*}
\textbf{E}=k\,\sum_{i=1}^{n}q_i\,\frac{\textbf{x}_i-\textbf{x}}{||\textbf{x}_i-\textbf{x}||^3}.
\end{equation*} 
O campo elétrico devido a um pequeno número de cargas pode ser calculado a partir do princípio da superposição linear. Mas se temos uma quantidade muito grande de cargas num determinado volume $V$, devemos calcular a \textit{densidade volumétrica de carga} $\rho$ num volume infinitesimal situado em $\textbf{x}_0$ e em seguida integrar sobre o volume $V$ para obter a quantidade total de carga $Q$. A densidade de carga é definida por
\begin{equation*}
\rho(\textbf{x}_0)=\lim_{\Delta V_i \to 0}\frac{\Delta q_i}{\Delta V_i}=\frac{d\,q}{d\,V},
\end{equation*}
medida, no SI, em $C/m^3$. A quantidade total de carga $Q=\sum_i \Delta\,q_i$ no volume $V$ é
\begin{equation*}
Q=\int_{V}\rho(\textbf{x}_0)\,dV.
\end{equation*}

O \textit{fluxo elétrico} é definido como a quantidade linhas do campo elétrico que atravessa uma dada superfície, e é dado pela equação
\begin{equation*}
\Phi_\textbf{E}=\textbf{E}\cdot\textbf{A}. 
\end{equation*} 
O \textit{vetor área} é definido como a magnitude da área da superfície atravessada apontando na direção do vetor normal, $\textbf{A}=A\,\textbf{n}$, e estamos considerando um campo elétrico uniforme $\textbf{E}$ que se desloca na direção $\textbf{n}$, ou seja, é perpendicular à superfície $A$ como podemos observar na figura \ref{fig.flux_ele}.
\begin{figure}[!htb]
\centering
\includegraphics[scale=.5]{campo_area}
\caption{\textit{Fluxo elétrico, linhas de campo elétrico passando através de uma superfície.}}
\label{fig.flux_ele}
\end{figure}
Mas se o campo elétrico se propaga formando um ângulo $\theta$ com o vetor normal da superfície, então o fluxo elétrico é dado por
\begin{equation*}
\Phi_\textbf{E}=\textbf{E}\cdot\textbf{A}=E\,A\,\cos\theta,
\end{equation*}
com $E=\textbf{E}\cdot\textbf{n}$ sendo a componente do campo elétrico na direção $\textbf{n}$. Em geral uma superfície pode ser curva e estamos interessados numa superfície \textit{fechada}, ou seja, aquela que engloba um determinado volume, o qual contém uma carga elétrica. Tomando uma área bem pequena dessa superfície, $\Delta\textbf{A}_i$, o campo elétrico pode ser variável em cada parte da superfície e nessas condições temos que o fluxo nessa pequena região é dado por
\begin{equation*}
\Delta\,\Phi_\textbf{E}=\textbf{E}_i\cdot\Delta\,\textbf{A}_i.
\end{equation*}
O fluxo positivo atravessando toda a superfície de dentro para fora é calculado tomando o limite quando $\Delta\textbf{A}_i\to 0$ e aumentando infinitamente a quantidade dessas pequenas áreas
\begin{equation}\label{eq.fluxo_eletr}
\Phi_\textbf{E}=\lim_{i\to\infty}\sum_i\textbf{E}_i\cdot\textit{d}\textbf{A}_i=\int\int_S\textbf{E}\cdot\textit{d}\textbf{A}.
\end{equation}

Considere uma carga pontual positiva $q$ localizada no centro de uma esfera imaginária de raio $r$, onde essa carga produz um campo elétrico que aponta na direção radial conforme a figura \ref{fig.esfe_gauss}. Sabemos que a área da superfície dessa esfera é dada por $A=4\pi\,r^2$ e que, segundo a equação \ref{eq.campo_eletrico}, a magnitude do campo elétrico em qualquer ponto da superfície esférica é
\begin{equation*}
E=\frac{q}{4\pi\epsilon_0\,r^2},
\end{equation*}
assim o fluxo elétrico é calculado usando a equação \ref{eq.fluxo_eletr}.
\begin{align*}
\Phi_\textbf{E}&=\int\int_S\textbf{E}\cdot\textit{d}\textbf{A}\\
&=\int\int_S\textbf{E}\cdot\textbf{n}\,\textit{d}A\\
&=E\,\int\int_S\textit{d}A\\
&=E\,A\\
&=\frac{q}{4\pi\epsilon_0\,r^2}\,4\pi\,r^2\\
&=\frac{q}{\epsilon_0}.
\end{align*}
Na demonstração acima escolhemos uma esfera como \textit{superfície Gaussiana} mas, introduzindo o conceito de \textit{ângulo sólido}, vemos que a demonstração é válida para qualquer superfície fechada, utilizada em aplicações que apresentem mais ou menos alguma simetria (esférica, planar ou cilíndrica). Para mais detalhes consultar \cite{jackson_classical_1999}. Assim, concluímos que o fluxo elétrico através de uma superfície fechada que apresente mais ou menos alguma simetria é diretamente proporcional à quantidade de carga enclausurada pela superfície. Matematicamente, a \textit{lei de Gauss} para o fluxo elétrico é
\begin{equation*}
\Phi_\textbf{E}=\int\int_S\textbf{E}\cdot\textit{d}\textbf{A}=\frac{q}{\epsilon_0}.
\end{equation*}
\begin{figure}[!htb]
\centering
\includegraphics[scale=.6]{esfera_gaussiana}
\caption{\textit{Esfera Gaussiana enclausurando uma carga positiva $q$. Nessas condições, o ângulo entre o vetor campo elétrico e o vetor normal à superfície infinitesimal $d\textbf{A}$ é zero.}}
\label{fig.esfe_gauss}
\end{figure}

Uma carga elétrica produz um campo elétrico, e de maneira similar uma barra magnética, ou ímã, produz um \textit{campo magnético} $\textbf{B}$. Um ímã possui um polo norte de onde partem as linhas de campo magnético e um polo sul por onde as linhas de campo magnético retornam ao ímã (figura \ref{fig.barras_mag}). Diferentemente das cargas elétricas que são observadas isoldamente na natureza, os dois polos magnéticos sempre aparecem aos pares, ou seja, monopolos magnéticos não existem isoladamente apesar de a suposição de sua existência ser de interesse teórico. Assim, sempre que um ímã é fracionado, mesmo que em partes muito elementares, o resultado sempre será um novo ímã com dois polos magnéticos conforme a figura \ref{fig.barras_mag}.
\begin{figure}[!htb]
\centering
\includegraphics[scale=.7]{barras_magn}
\caption{\textit{Barras magnéticas onde polos de mesmo sinal se repelem e polos de sinais contrários se atraem.}}
\label{fig.barras_mag}
\end{figure}
Como não existem monopolos magnéticos, o campo magnético deve ser definido de forma diferente do campo elétrico, e experimentalmente foram observadas algumas características relacionadas ao movimento de uma carga elétrica $q$ com velocidade $\textbf{v}$ num campo magnético $\textbf{B}$:
\begin{itemize}
\item a magnitude da força magnética $\textbf{F}_B$ é proporcional à $v$, $B$ e $q$, onde $v$ e $B$ são as magnitudes da velocidade e do campo magnético respectivamente,
\item a direção de $\textbf{F}_B$ é perpendicular ao plano formado por $\textbf{v}$ e $\textbf{B}$,
\item $\textbf{F}_B$ é proporcional ao $\sin\theta$, o ângulo formado por $\textbf{v}$ e $\textbf{B}$. Se $\textbf{v}$ e $\textbf{B}$ são paralelos então $\textbf{F}_B=0$, e
\item o sentido de $\textbf{F}_B$ depende do sinal da carga $q$.
\end{itemize}
Essas observações são ilustradas na figura \ref{fig.froca_mag_veloc} e a força magnética é definida como
\begin{equation*}
\textbf{F}_B=q\textbf{v}\times\textbf{B}.
\end{equation*}
\begin{figure}[!htb]
\centering
\includegraphics[scale=.5]{forca_camp_mag_veloc}
\caption{\textit{Força magnética agindo numa carga elétrica que se desloca num campo magnético.}}
\label{fig.froca_mag_veloc}
\end{figure}
Teoricamente, poderíamos tentar determinar a lei de Gauss para o fluxo magnético com o mesmo procedimento aplicado ao fluxo elétrico e obter
\begin{equation*}
\Phi_\textbf{B}=\int\int_S\textbf{B}\cdot\textit{d}\textbf{A}=\frac{q_m}{\mu_0},
\end{equation*} 
onde $q_m$ é a carga magnética (suposto monopolo magnético) enclausurado pela superfície Gaussiana, $B$ é o campo magnético e $\mu_0$ é a \textit{permeabilidade magnética no vácuo}. No entanto, não foi constatada a existência de qualquer carga magnética isolada mesmo após muitos esforços. Como $q_m=0$, temos que a lei de Gauss para o magnetismo é
\begin{equation}\label{eq.gauss_flux_mag}
\Phi_\textbf{B}=\int\int_S\textbf{B}\cdot\textit{d}\textbf{A}=0.
\end{equation}
Conforme podemos ver na figura tal, a equação \ref{eq.gauss_flux_mag} implica que a quantidade de linhas do campo magnético saindo da superfície é igual à quantidade que está entrando, ou seja, não há uma origem isolada e um término isolado para o fluxo magnético como há para o fluxo elétrico. Outro problema é que a barra imantada atravessa a superfície que, de acordo com a hipóteses da lei de Gauss, deveria ser fechada.
\begin{figure}[!htb]
\centering
\includegraphics[scale=.3]{flux_ele_mag}
\caption{\textit{As linhas do campo magnético que emanam do polo norte do ímã em direção ao polo sul retornam para dentro da superfície Gaussiana descrevendo um laço fechado.}}
\label{fig.flux_elet_magn}
\end{figure}




\section{Equações de Maxwell}

\section{Generalizações da teoria}

\section{Conclusões}
\chapter{Fundamentos de Elasticidade}\label{sec.fund_elast}

\section{Introdu\c{c}\~ao}
A teoria formal da propaga\c{c}\~ao de ondas s\'ismicas repousa nas intera\c{c}\~oes entre as particulas infinitesimais discretas do meio \`a medida que uma deforma\c{c}\~ao se propaga. \'E muito dif\'icil estudar individualmente cada uma dessas intera\c{c}\~oes, mas dados experimentais que foram coletados como resultados dessas intera\c{c}\~oes sugerem que as mesmas podem ser consideradas em conjunto. Assim, o estudo da propaga\c{c}\~ao de ondas s\'ismicas atrav\'es de camadas de subsuperf\'icie num material discretizado pode ser feito considerando o meio como cont\'inuo, e tais estudos s\~ao os objetos da \textit{mec\^anica do cont\'inuo}. 

No desenvolvimento te\'orico da mec\^anica do cont\'inuo nao s\~ao consideradas as caracter\'isticas at\^omicas da mat\'eria bem como as intera\c{c}\~oes entre essas part~\'iculas, ou seja, a mat\'eria n\~ao \'e estudada do ponto de vista microsc\'opico. Segundo \cite{slawinski}, tal abordagem se justifica pelo fato de que a mat\'eria \'e formada por part\'iculas suficientemente pouco espa\c{c}adas e suas caracter\'isticas e comportamento podem ser descritos por fun\c{c}\~oes cont\'inuas e diferenci\'aveis. Assim, \'e assummido que elementos infinitesimais da mat\'eria t\^em as mesmas propriedades observadas em experimentos macrosc\'opicos, pois essa hip\'otese permite a cria\c{c}\~ao de um modelo matem\'atico abstrato \textit{efetivo} na descri\c{c}\~ao da realidade f\'isica. Como exemplo, vamos considerar a cor de um objeto. Pr\'otons e el\'etrons n\~ao possuem cor, mas os meios materiais (que s\~ao formados por pr\'otons e el\'etrons) t\^em a capacidade de absorver ou refletir determinados comprimentos de ondas eletromagn\'eticas as quais determinam a cor de cada meio. Outros conceitos da mec\^anica do cont\'inuo sao elasticidade, viscosidade, fric\c{c}\~ao, rigidez, etc, como veremos mais a frente.


\section{Fatos experimentais}
A teoria sobre elasticidade est\'a baseada em conceitos primitivos e conclus\~oes estabelecidas a partir de fatos experimentais verificados em v\'arios textos sobre o assunto como \cite{liu}, \cite{dahlem} e \cite{slawinski}. Adicionalmente, em geral as equa\c{c}\~oes que governam a propaga\c{c}\~ao de ondas em meios el\'asticos s\~ao n\~ao-lineares. Contudo, em experimentos s\'ismicos foi constatado que aspectos importantes da propaga\c{c}\~ao de ondas podem ser analisados a partir de equa\c{c}\~oes lineares, resultando numa abordagem chamada \textit{teoria da elasticidade linearizada}.

\subsection{Deforma\c{c}\~ao}

A \textit{deforma\c{c}\~ao} de um meio el\'astico cont\'inuo \'e a mudan\c{c}a na posi\c{c}\~ao dos pontos que comp\~oem o corpo em rela\c{c}\~ao uns aos outros. Ou seja, h\'a uma mudan\c{c}a relativa entre os pontos e n\~ao um deslocamento do corpo como um todo e sem mudan\c{c}a de sua forma, caso em que ter\'iamos um \textit{movimento r\'igido}. Nesta subse\c{c}\~ao estamos interessados nas caracter\'isticas geom\'etricas relativas a deforma\c{c}\~ao de um corpo. N\~ao estamos considerando as causas de deforma\c{c}\~ao de um corpo, como aplica\c{c}\~ao de carga ou varia\c{c}\~ao de temperatura, nem discutiremos a composi\c{c}\~ao do material, assumindo apenas que o mesmo seja cont\'inuo e el\'astico. Assim, vamos relacionar as caracter\'isticas geom\'etricas de um corpo antes da deforma\c{c}\~ao com as caracter\'isticas ap\'os a deforma\c{c}\~ao.

\subsection{Dedu\c{c}\~ao do Tensor de Deforma\c{c}\~oes}

Para determinar o tensor de deforma\c{c}\~oes vamos considerar dois pontos pertencentes ao espaco $\mathbb{R}^3$ bastante pr\'oximos um do outro denotados por
\begin{align*}
\mathbf{x}&=(x_1,x_2,x_3)\\
\mathbf{y}=\mathbf{x}+d\mathbf{s}&=(x_1+dx_1,x_2+dx_2,x_3+dx_3).
\end{align*}
O quadrado da dist\^ancia entre esses dois pontos \'e
\begin{equation}\label{eq.dist_antes_defor}
\norm{d\mathbf{s}}^2=(dx_1)^2+(dx_2)^2+(dx_3)^2.
\end{equation}
A aplica\c{c}\~ao de uma deforma\c{c}\~ao depende do ponto de aplica\c{c}\~ao, ou seja, a deforma\c{c}\~ao aplicada no ponto $\mathbf{x}$ difere da aplica\c{c}\~ao no ponto $\mathbf{y}$. Caso o vetor que d\'a a deforma\c{c}\~ao tenha componentes constantes, n\~ao teremos uma deforma\c{c}\~ao relativa, apenas um transla\c{c}\~ao dos pontos. Assim, podemos definir o \textit{vetor de deslocamento} para cada ponto de aplica\c{c}\~ao
\begin{align*}
\mathbf{u}&=(u_1,u_2,u_3)\\
\mathbf{v}&=(v_1,v_2,v_3),
\end{align*}
e som\'a-los aos respectivos pontos $\mathbf{x}$ e $\mathbf{y}$ para obter suas posi\c{c}\~oes ap\'os a deforma\c{c}\~ao,
\begin{align*}
\mathbf{x}^*&=(x_1+u_1,x_2+u_2,x_3+u_3)\\
\mathbf{y}^*&=(x_1+dx_1+v_1,x_2+dx_2+v_2,x_3+dx_3+v_3).
\end{align*}
Subtraindo, obtemos o vetor que d\'a a diferen\c{c}a entre os pontos ap\'os a deforma\c{c}\~ao
\begin{equation}\label{eq.dist_apos_defor}
d\mathbf{s}^*=(dx_1+v_1-u_1,dx_2+v_2-u_2,dx_3+v_3-u_3).
\end{equation}
Como a varia\c{c}\~ao entre os pontos $\mathbf{x}$ e $\mathbf{y}$ \'e infinitesimal, vamos aplicar a expans\~ao de Taylor de segunda ordem em torno do ponto $\mathbf{x}$ e desprezar o resto de Lagrange para escrever as componentes de $\mathbf{v}$ em fun\c{c}\~ao das componentes de $\mathbf{u}$, aproximadamente,
\begin{align*}
v_1&\approx u_1+\frac{\partial u_1}{\partial x_1}\Bigg\vert_{\mathbf{x}}dx_1+\frac{\partial u_1}{\partial x_2}\Bigg\vert_{\mathbf{x}}dx_2+\frac{\partial u_1}{\partial x_3}\Bigg\vert_{\mathbf{x}}dx_3\\\\
v_2&\approx u_2+\frac{\partial u_2}{\partial x_1}\Bigg\vert_{\mathbf{x}}dx_1+\frac{\partial u_2}{\partial x_2}\Bigg\vert_{\mathbf{x}}dx_2+\frac{\partial u_2}{\partial x_3}\Bigg\vert_{\mathbf{x}}dx_3\\\\
v_3&\approx u_3+\frac{\partial u_3}{\partial x_1}\Bigg\vert_{\mathbf{x}}dx_1+\frac{\partial u_3}{\partial x_2}\Bigg\vert_{\mathbf{x}}dx_2+\frac{\partial u_3}{\partial x_3}\Bigg\vert_{\mathbf{x}}dx_3.
\end{align*}
Substituindo esses valores na equa\c{c}\~ao \ref{eq.dist_apos_defor}, simplificando e introduzindo a nota\c{c}\~ao de somat\'orio temos
\begin{equation*}
d\mathbf{s}^*\approx\left(dx_1+\sum_{i=1}^3\frac{\partial u_1}{\partial x_i}\Bigg\vert_{\mathbf{x}}dx_i\,,\,dx_2+\sum_{i=1}^3\frac{\partial u_2}{\partial x_i}\Bigg\vert_{\mathbf{x}}dx_i\,,\,dx_3+\sum_{i=1}^3\frac{\partial u_3}{\partial x_i}\Bigg\vert_{\mathbf{x}}dx_i\right).
\end{equation*}
O quadrado da dist\^ancia entre os pontos ap\'os a deforma\c{c}\~ao \'e dado por
\begin{equation*}
\norm{d\mathbf{s}^*}^2\approx\left(dx_1+\sum_{i=1}^3\frac{\partial u_1}{\partial x_i}\Bigg\vert_{\mathbf{x}}dx_i\right)^2+\left(dx_2+\sum_{i=1}^3\frac{\partial u_2}{\partial x_i}\Bigg\vert_{\mathbf{x}}dx_i\right)^2+\left(dx_3+\sum_{i=1}^3\frac{\partial u_3}{\partial x_i}\Bigg\vert_{\mathbf{x}}dx_i\right)^2.
\end{equation*}
Abrindo cada uma das parcelas quadr\'aticas, temos
\begin{align*}
\norm{d\mathbf{s}^*}^2&\approx(dx_1)^2+2\,dx_1\,\sum_{i=1}^3\frac{\partial u_1}{\partial x_i}\Bigg\vert_{\mathbf{x}}dx_i+\left(\sum_{i=1}^3\frac{\partial u_1}{\partial x_i}\Bigg\vert_{\mathbf{x}}dx_i\right)^2\\\\
&+(dx_2)^2+2\,dx_2\,\sum_{i=1}^3\frac{\partial u_2}{\partial x_i}\Bigg\vert_{\mathbf{x}}dx_i+\left(\sum_{i=1}^3\frac{\partial u_2}{\partial x_i}\Bigg\vert_{\mathbf{x}}dx_i\right)^2\\\\
&+(dx_3)^2+2\,dx_3\,\sum_{i=1}^3\frac{\partial u_3}{\partial x_i}\Bigg\vert_{\mathbf{x}}dx_i+\left(\sum_{i=1}^3\frac{\partial u_3}{\partial x_i}\Bigg\vert_{\mathbf{x}}dx_i\right)^2.
\end{align*}
Pela equa\c{c}\~ao \ref{eq.dist_antes_defor}, a coluna  da esquerda \'e $\norm{d\mathbf{s}}^2$. Como estamos trabalhando com quantidades infinitesimais, podemos negligenciar a coluna da direita por se tratar do quadrado do gradiente de cada componente do vetor de deslocamento num produto escalar com o vetor que d\'a a dist\^ancia entre os pontos $\mathbf{x}$  e $\mathbf{y}$. Assim,
\begin{equation*}
\norm{d\mathbf{s}^*}^2\approx\norm{d\mathbf{s}}^2+\sum_{i=1}^3\sum_{j=1}^3\left(\frac{\partial u_i}{\partial x_j}\Bigg\vert_{\mathbf{x}}+\frac{\partial u_j}{\partial x_i}\Bigg\vert_{\mathbf{x}}\right)dx_i\,dx_j,
\end{equation*}
onde o termo entre par\^enteses \'e definido como o \textit{tensor de deforma\c{c}\~ao} na teoria da elasticidade,
\begin{equation*}
\varepsilon_{i,j}=\frac{1}{2}\left(\frac{\partial u_i}{\partial x_j}\Bigg\vert_{\mathbf{x}}+\frac{\partial u_j}{\partial x_i}\Bigg\vert_{\mathbf{x}}\right),\qquad\text{onde}\qquad i,j\,\in\,\{1,2,3\}.
\end{equation*}








\section{Equações de Lamé}

\section{Generalizações da teoria}

\section{Conclusões}
\chapter{Acoplamento Magneto-elástico}

\section{Introdução}

\section{Modelo de Dunkin e Erigen}

\section{Conclusões}
\chapter{Recondicionamento do Modelo de Dunkin e Erigen}\label{sec.recon_model_dun_eri}
Uma das contribui\c{c}\~oes dessa monografia \'e a utiliza\c{c}\~ao de hip\'oteses simplificadoras de ordem f\'isica e experimental, dispon\'iveis na literatura, para reeescrever o modelo do acoplamento magneto-el\'astico de forma que o mesmo possa receber um tratamento matem\'atico e computacional, no sentido de solucionar um problema direto.

Como vimos na subse\c{c}\~ao \ref{sec.magnetizacao_polarizacao}, a polariza\c{c}\~ao e magnetiza\c{c}\~ao de um determinado material depende das caracter\'isticas de cada material. De acordo com \cite{jackson_classical_1999} e \cite{griffiths}, as equa\c{c}\~oes constitutivas apresentadas na subse\c{c}\~ao \ref{sec.constitutivas_dunkin} podem n\~ao ser simples pois existe uma diversidade enorme de propriedades el\'etricas e magn\'eticas dos materias, especialmente em s\'olidos cristal\'inicos  e materias ferroel\'etricos e ferromagn\'eticos que t\^em polariza\c{c}\~ao e magnetiza\c{c}\~ao n\~ao nulos mesmo na absten\c{c}\~ao de aplica\c{c}\~ao de campos eletromagn\'eticos. Com excess\~ao desses tipos de materias, a aplica\c{c}\~ao de campos eletromagn\'eticos produzem polariza\c{c}\~ao e magnatiza\c{c}\~ao proporcional aos campos aplicados, e a rela\c{c}\~ao do campo de densidade de fluxo el\'etrico com o campo el\'etrico bem como a rela\c{c}\~ao do campo magn\'etico auxiliar com o campo magn\'etico s\~ao consideradas lineares, pois a contribui\c{c}\~ao das parcelas n\~ao-lineares tornam-se desprez\'iveis. Podemos ainda considerar a permeabilidade magn\'etica de muitos materiais tendo valor muito pr\'oximo da permeabilidade magn\'etica no v\'acuo, assim $\mu=\mu_0$. Com isso, temos que o escalar $\alpha$ definido em \ref{eq.constitutiva_1} tem seu valor considerado nulo, e as express\~oes para o campo de densidade de fluxo el\'etrico e o campo magn\'etico da subse\c{c}\~ao \ref{sec.constitutivas_dunkin} se tornam
\begin{align}\label{eq.constitutiva_alpha_1}
\mathbf{D}&=\epsilon\,\mathbf{E}\\\nonumber\\\label{eq.constitutiva_alpha_2}
\mathbf{B}&=\mu\,\mathbf{H}.
\end{align}
Substituindo a equa\c{c}\~ao \ref{eq.constitutiva_alpha_1} na equa\c{c}\~ao \ref{eq.campo_dunkin_1} e aplicando a transformada de Fourier, temos a rela\c{c}\~ao entre o campo el\'etrico e o campo magn\'etico auxiliar dada por
\begin{equation}
\nabla\times\mathbf{\widehat{E}}=i\,\omega\,\mu_0\mathbf{\widehat{H}},
\end{equation}
onde $i$ \'e um n\'umero complexo, $\omega$ \'e a frequ\^encia temporal e a nota\c{c}\~ao $\widehat{\,\,}$ significa que a fun\c{c}\~ao vetorial est\'a no dom\'inio da frequ\^encia temporal.

Ondas eletromagn\'eticas se propagam com velocidade da luz que \'e limitada. Segundo \cite{jackson_classical_1999}, num sistema onde as dimens\~oes s\~ao pequenas comparadas ao comprimento de onda eletromagn\'etica e comparadas \`a escala de tempo dominante, podemos tratar a velocidade da luz como instant\^anea num regime denominado \textit{quasi-estacion\'ario}. Como consequ\^encia dessa premissa, em meios condutivos a contribui\c{c}\~ao do campo de densidade de fluxo el\'etrico \'e muito pequena quando comparada \`a contribui\c{c}\~ao da densidade de corrente el\'etrica na produ\c{c}\~ao de campos magn\'eticos. Assim, podemos desprezar a parcela referente \`a corrente deslocada introduzida por Maxwell na lei de Amper\`e, o que implica (pela equa\c{c}\~ao \ref{eq.campo_dunkin_3}) em $\rho_e=0$. 
Vamos supor ainda que o campo magn\'etico auxiliar medido durante o efeito magneto-el\'astico seja uma combina\c{c}\~ao do campo magn\'etico gerado $\mathbf{H}$ mais o campo magn\'etico natural da Terra $\mathbf{H}^0$, e para manter a nota\c{c}\~ao vamos usar a substitui\c{c}\~ao 
\begin{equation}
\mathbf{H}\longrightarrow\mathbf{H}+\mathbf{H}^0.
\end{equation}
Utilizando a equa\c{c}\~ao \ref{eq.constitutiva_alpha_2}, o campo de densidade de corrente el\'etrica dado pela subse\c{c}\~ao \ref{sec.constitutivas_dunkin} poder ser reescrito como
\begin{equation}
\mathbf{J}=\sigma\,\mathbf{E}+\mathbf{v}\times\sigma\,\mu_0\mathbf{H}.
\end{equation}
Substituindo esta \'ultima rela\c{c}\~ao juntamente com a equa\c{c}\~ao \ref{eq.constitutiva_alpha_1} na equa\c{c}\~ao \ref{eq.campo_dunkin_2}, e aplicando a transformada de Fourier, temos
\begin{equation}
\nabla\times\mathbf{\widehat{H}}=(\sigma-i\,\epsilon\,\omega)\,\mathbf{\widehat{E}}+\mathbf{\widehat{v}}\times\sigma\mu_0\mathbf{H}^0,
\end{equation}
onde $\mathbf{\widehat{v}}=-i\,\omega\mathbf{\widehat{u}}$ \'e a velocidade de deslocamento do meio. Na dedu\c{c}\~ao da equa\c{c}\~ao acima, estamos considerando que a contribui\c{c}\~ao da parcela $\mathbf{\widehat{v}}\times\sigma\mu_0\mathbf{\widehat{H}}$ \'e desprez\'ivel se comparada \`a contribui\c{c}\~ao do campo geomagn\'etico, ainda como uma consequ\^encia do regime quasi-estacion\'ario.

De acordo com \cite{Knopoff_1955}, a altera\c{c}\~ao que o campos eletromagn\'eticos aplicam em ondas el\'asticas \'e desprez\'ivel, e assim podemos excluir a for\c{c}a de Lorentz e reescrever a equa\c{c}\~ao \ref{eq.campo_dunkin_5} no dom\'inio da frequ\^encia na forma matricial como 
\begin{equation}
-i\,\omega\rho\,\mathbf{\widehat{v}}=\nabla\cdot\widehat{\tau} + \mathbf{\widehat{F}}.
\end{equation}
A lei de Hooke dada na subse\c{c}\~ao \ref{sec.constitutivas_dunkin} pode ser reescrita no dom\'inio da frequ\^encia e em sua forma matricial como
\begin{equation}
\widehat{\tau}=\lambda\,\nabla\cdot\mathbf{\widehat{u}}\cdot\,I + \mu\,(\nabla\,\mathbf{\widehat{u}}+\nabla\mathbf{\widehat{u}}^\top),
\end{equation}
onde $I$ \'e a matriz identidade e $\nabla\mathbf{\widehat{u}}=(\nabla u_1,\nabla u_2,\nabla u_3)$ \'e o gradiente do campo vetorial que d\'a o deslocamento do meio de propaga\c{c}\~ao das ondas.

Substituindo a equa\c{c}\~ao \ref{eq.constitutiva_alpha_2} na equa\c{c}\~ao \ref{eq.campo_dunkin_4} e aplicando a transformada de Fourier, temos
\begin{equation}
\nabla\cdot\mathbf{\widehat{H}}=0.
\end{equation}

Assumindo as hip\'oteses simplificadoras acima, com a depend\^encia do tempo dada por $e^{(-i\,\omega\,t)}$, as equa\c{c}\~oes diferenciais parciais linearizadas de magneto-elasticidade s\~ao
\begin{align*}
\nabla\times\mathbf{\widehat{E}}&=i\,\omega\,\mu_0\mathbf{\widehat{H}}\\\\
\nabla\times\mathbf{\widehat{H}}&=(\sigma-i\,\epsilon\,\omega)\,\mathbf{\widehat{E}}+\mathbf{\widehat{v}}\times\sigma\mu_0\mathbf{H}^0\\\\
-i\,\omega\rho\,\mathbf{\widehat{v}}&=\nabla\cdot\widehat{\tau} + \mathbf{\widehat{F}}\\\\
\widehat{\tau}&=\lambda\,\nabla\cdot\mathbf{\widehat{u}}\cdot\,I + \mu\,(\nabla\,\mathbf{\widehat{u}}+\nabla\mathbf{\widehat{u}}^\top)\\\\
\nabla\cdot\mathbf{\widehat{H}}&=0
\end{align*}

Vamos definir $\sigma^*=(\sigma-i\,\epsilon\,\omega)$. No subsolo, por conta do regime quasi-estacion\'ario, $(\sigma>>\epsilon\,\omega)$  e  temos $\sigma^*=\sigma$. No ar, a condutividade \'e zero e a permeabilidade el\'etrica \'e pr\'oxima a do v\'acuo, assim temos $\sigma^*=-i\,\epsilon_0\omega$.
\chapter{Conclusões e Trabalhos Futuros}

Muitas pesquisas v\^em sendo realizadas no sentido de efetuar simula\c{c}\~oes num\'erico-computacionais que possam descrever diversos fen\^omenos f\'isicos relacionados \`a prospec\c{c}\~ao de petr\'oleo ou outro bem mineral, assim como fen\^omenos f\'isicos relacionados a terremotos ou que se aplicam a outros objetos de estudo. Essas simula\c{c}\~oes s\~ao ainda confrontadas com experimentos de campo na busca por consist\^encia entre essas duas faces do desenvolvimento de uma teoria. Numa oportunidade futura vamos desenvolver de forma anal\'itico-matem\'atica as EDP's da magneto-elasticidade,  e em seguida criar um algoritmo computacional capaz de efetuar simula\c{c}\~oes que nos auxiliem a estudar o efeito magneto-el\'astico, bem como entender e expandir a teoria geral que trata da intera\c{c}\~ao entre mec\^anica do cont\'inuo e eletromagnetismo em explora\c{c}\~ao de petr\'oleo.

O efeito magneto-el\'astico \'e descrito matematicamente pelo conjunto de EDP's formado pelo sistema de Maxwell e pelo sistema de Lam\`e, os quais s\~ao utilizados no estudo da propaga\c{c}\~ao acoplada de ondas s\'ismicas e eletromagn\'eticas na subsuperf\'icie terrestre. O acoplamento foi caracterizado, primeiramente, pela varia\c{c}\~ao que a for\c{c}a de Lorentz provoca no deslocamento do meio condutivo, simbolizado pela adi\c{c}\~ao da parcela referente \`a esta for\c{c}a na equa\c{c}\~ao do movimento de Cauchy. Segundo, pela a altera\c{c}\~ao eletromagn\'etica gerada pela passagem de uma onda s\'ismica que faz um meio condutivo oscilar no campo geomagn\'etico, simbolizada pela adi\c{c}\~ao desta varia\c{c}\~ao \`a lei de Amp\`ere-Maxwell. Estamos estudando o caso parcialmente acoplado no espaco 3D, mas desejamos analisar tamb\'em o acoplamento total considerando tanto o espa\c{c}o 3D como o espa\c{c}o 1D, esperando que os desenvolvimentos e resultados em cada caso possam se complementar mutuamente.

Algumas hip\'oteses de ordem f\'isica, como o regime quasi-estacion\'ario por exemplo, foram necess\'arias para simplificar o modelo, linearizando as equa\c{c}\~oes e possibilitando o desenvolvimento anal\'itico das mesmas. Sendo assim, buscaremos pelas solu\c{c}\~oes dessas EDP's raciocinando basicamente com duas alternativas. Uma delas \'e trasnsformar as EDP's em EDO's utilizando ferramentas como as transformadas laterais de Fourier, transformadas de Hankel e mudan\c{c}as de eixos coordenados, escrevendo as equa\c{c}\~oes num formato onde \'e poss\'ivel aplicar um metodo matricial espec\'ifico para estudo de propaga\c{c}\~ao de ondas em meios estratigr\'aficos. Outra alternativa \'e reescrever as equa\c{c}\~oes em coordenadas cil\'indricas e fazer uso da hip\'otese de isotropia do meio de propaga\c{c}\~ao para reduzir as dimens\~oes do problema e obter as EDO's nas quais o m\'etodo matricial \'e aplicado.
\\
Neste trabalho apresentamos um tratamento matem\'atico das EDP's do efeito magneto-el\'astico encontrado em \cite{pinho_2018} , no sentido de propiciar a contru\c{c}\~ao de um algoritmo num\'erico est\'avel que possa descrever a propaga\c{c}\~ao acoplada de ondas el\'asticas e eletromagn\'eticas. Nesse tratamento foi fundamental a aplica\c{c}\~ao de conhecimentos da F\'isica-Matem\'atica, Geof\'isica e, em particular, um metodo matricial que facilita a an\'alise de propaga\c{c}\~ao de ondas em meios estratificados.

Vimos na subse\c{c}\~ao \ref{sec.matricial_poroelast} a possibilidade de an\'alise de dispers\~ao e de atenua\c{c}\~ao de ondas para casos diversos, onde tal an\'alise auxilia na verifica\c{c}\~ao e constru\c{c}\~ao de um c\'odigo computacional efetivo para descrever a propaga\c{c}\~ao dessas ondas. Numa oportuinidade futura, queremos aplicar a an\'alise de atenua\c{c}\~ao e dispers\~ao nesse sistema de EDP's do efeito magneto-el\'astico com a finalidade de ajudar a estudar o comportamento da propaga\c{c}\~ao.

Numa determinada abordagem, a an\'alise de casos mais simples auxilia no estudo de casos mais sofisticados. Por tanto, no intuito ainda de otimizar o estudo da propaga\c{c}\~ao das ondas, faremos o tratamento matem\'atico das EDP's de magneto-elasticidade para o caso unidimensional, considerando a propaga\c{c}\~ao em fun\c{c}\~ao do tempo e em fun\c{c}\~ao da profundidade. Neste caso podemos utilizar o m\'etodo matricial e a an\'alise de atenua\c{c}\~ao e dispers\~ao das ondas, e economizamos a utiliza\c{c}\~ao de transformadas e mudan\c{c}a de eixos coordenados.

O formato final das EDO's dado no cap\'itulo \ref{sec.trans_edp_2_edo} apresentou algumas vari\'aveis incluidas como fonte, diferentemente do que \'e preconizado por Ursin, onde todas a vari\'aveis devem estar inseridas no vetor $\mathbf{\Phi}$. Assim, analisaremos a possibilidade da aplica\c{c}\~ao de fun\c{c}\~oes de Green juntamente com o m\'etodo matricial para contornar esse problema. \'E poss\'ivel que essa abordagem traga desafios computacionais consider\'aveis e da\'i estudaremos tamb\'em outras alternativas. Uma delas \'e considerar o efeito magento-el\'astico para o caso totalmente acoplado e verificar se o novo formato das equa\c{c}\~oes permite a exclus\~ao de vari\'avies dadas como fonte. Outra possibilidade \'e escrever as equa\c{c}\~oes em coordenadas cil\'indricas, considerar as propriedades de isotropia das camadas e substituir as coordenadas horizontais somente pelo raio.

A implementa\c{c}\~ao do algoritmo computacional ser\'a realizada em linguagem C++, por conta de algumas caracter\'isticas apresentadas por esta linguagem descritas em \cite{bueno_2015}, como: ser de prop\'osito geral podendo ser utilizada na constru\c{c}\~ao de programas computacionais, aplicativos de sistemas embarcados e em computa\c{c}\~ao cient\'ifica; ser de alto n\'ivel e orientada a objeto, permitindo a propagama\c{c}\~ao simult\^anea realizada por v\'arios programadores trabalhando num mesmo projeto; fortemente tipada o que ajuda na detec\c{c}\~ao de \textit{bugs} e controle e gerenciamento de mem\'oria; ser a mais utilizada em sistemas complexos e grandes no uso de programa\c{c}\~ao paralela.

\bibliographystyle{plainnat}
%\bibliography{referencias}

\begin{thebibliography}{25}
\providecommand{\natexlab}[1]{#1}
\providecommand{\url}[1]{\texttt{#1}}
\expandafter\ifx\csname urlstyle\endcsname\relax
  \providecommand{\doi}[1]{doi: #1}\else
  \providecommand{\doi}{doi: \begingroup \urlstyle{rm}\Url}\fi


\bibitem[Cukavac(2008)]{Cukavac_2008}
M.~S. Cukavac.
\newblock Seismomagnetic insvestigations in kopaonik area.
\newblock \emph{MGB}, 2008.

\bibitem[Eringen(1963)]{eringen_1963}
J.W.~Dunkin e~A.C.~Eringen.
\newblock On the propagation of waves in an electromagnetic elastic solid.
\newblock \emph{International Journal of Engineering Science}, 1, 1963.


\bibitem[Tromp(1998)]{dahlem}
F.~A.~Dahlem e~J.~Tromp.
\newblock \emph{Theoretical Global Seismology}.
\newblock Princeton University Press, 1998.

\bibitem[Soboleva(1997)]{Mikhailenko_1997}
B.~G.~Mikhailenko e~O.~N.~Soboleva.
\newblock Mathematical modeling of seismomagnetic efects arising in the seismic
  wave motion in the earth's constant magnetic field.
\newblock \emph{Applied Mathematics Lettures}, 10\penalty0 (3):\penalty0
  47--51, 1997.

\bibitem[Pilipenko(1997)]{surkov_97}
V.~V.~Surkov e~V.~A.~Pilipenko.
\newblock Magnetic effects due to earthquakes and underground explosions: a
  review.
\newblock \emph{Annali di Geofisica}, XL\penalty0 (2):\penalty0 227--239, 1997.

\bibitem[Eringen(1962)]{Eringen_1962}
A.~C. Eringen.
\newblock \emph{Nonlinear Theory of Continuous Media}.
\newblock McGraw-Hill New York, 1962.

\bibitem[Griffiths(1999)]{griffiths}
D.~J. Griffiths.
\newblock \emph{Introduction to Electrodynamics}.
\newblock Prentice-Hall, 1999.

\bibitem[Guglielmi(1986a)]{guglielmi_86a}
A.~V. Guglielmi.
\newblock Magnetoelastic waves.
\newblock \emph{Izv. Akad. Nauk SSSR, Fizika Zemli}, 7\penalty0 (112), 1986a.

\bibitem[Guglielmi(1986b)]{guglielmi_86b}
A.~V. Guglielmi.
\newblock Excitation of oscillations of the electromagnetic field by elastic
  waves in the conducting body.
\newblock \emph{Geomagn. Aeron.}, 27\penalty0 (3):\penalty0 467--470, 1986b.

\bibitem[Jackson(1999)]{jackson_classical_1999}
J.~D. Jackson.
\newblock \emph{Classical electrodynamics}.
\newblock Wiley, New York, {NY}, 3rd ed., 1999.


\bibitem[Knopoff(1955)]{Knopoff_1955}
E.~L. Knopoff.
\newblock The interaction between elastic waves motions and magnetic field in
  electrical conductors.
\newblock \emph{J. Geophys. Res.}, 60\penalty0 (4):\penalty0 617--629, 1955.

\bibitem[Yerzhanov(1985)]{yerzhanov_85}
T.~E. Nasynbaev e A. V.~Bushuev L.~S.~Yerzhanov, A. K.~Kurskeev.
\newblock Geomagnetics observations during the massa experiments.
\newblock \emph{Izv. Akad. Nauk SSSR, Fizika Zemli}, 11:\penalty0 80--82, 1985.

\bibitem[Liu(2002)]{liu}
I.~S. Liu.
\newblock \emph{Continuum Mechanics}.
\newblock Spring-Verlag, Berlim-Heidelberg, 2002.

\bibitem[Sadovsky(1980)]{Sadovsky_1980}
M.A. Sadovsky.
\newblock Electro magnetic precursors of earthquakes.
\newblock \emph{Dokl. Acad. Nauka}, 1980.

\bibitem[Slawinski(2007)]{slawinski}
M.~A. Slawinski.
\newblock \emph{Waves And rays in elastic continua}.
\newblock World Scientific Publishing Company, 2 ed, 2007.


\bibitem[Sommerfeld(1952)]{sommerfeld_52}
A.~Sommerfeld.
\newblock Electrodynamics.
\newblock \emph{Academic Press}, 1952.

\bibitem[Stacey(1964)]{stacey_64}
F.~D. Stacey.
\newblock Seismo-magnetic effect.
\newblock \emph{Pure Applied Geophysics}, 58\penalty0 (11):\penalty0 5--23,
  1964.

\bibitem[Surkov(1989a)]{surkov_89a}
V.~V. Surkov.
\newblock Local changes in geomagnetics and geoeletrics fields under rocks
  deformation near the earth surface.
\newblock \emph{Izv. Akad. Nauk SSSR, Fizika Zemli}, 5:\penalty0 91--96, 1989a.

\bibitem[Surkov(1989b)]{surkov_89b}
V.~V. Surkov.
\newblock Distortion of external magnetic field by a longitudinal acoustic
  wave.
\newblock \emph{Magnetic Hydro-Dynamics}, 2:\penalty0 9--12, 1989b.

\bibitem[Anisimov(1985)]{Anisimov_1985}
E.A. Ivanov M.V. Pedanov N.N.Rusakov V.A. Troizhkya e V.E.~Goncharov
  S.V.~Anisimov, M.B.~Gokhberg.
\newblock Short period oscillations of electromagnetic field of the earth after
  explosion.
\newblock \emph{Dokl. Acad. Nauka}, 281\penalty0 (3):\penalty0 556--559, 1985.

\bibitem[Rikitake(1980)]{Rikitake_80}
H.~Tanaka N. Ohshiman Y. Sasai Y. Ishikawa S. Koyama M. Kawamura e K.~Ohchi
  T.~Rikitake, Y.~Honkura.
\newblock Changes in the geomagnetic field associated with earthquakes in the
  izu peninsula, japan.
\newblock \emph{J. Geomag. Geoelectr.}, 32:\penalty0 721--739, 1980.

\end{thebibliography}

\end{document}