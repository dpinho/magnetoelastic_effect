\chapter{Introdução}
Esta monografia tem como finalidade principal o detalhamento da teoria que envolve o acoplamento de ondas eletromagnéticas e ondas elásticas que se propagam em meios estratificados, e homogêneos por camada, no subsolo terrestre. O desenvolvimento dessa teoria segue o modelo apresentado por \cite{erigen_1963} que trata da propagação de ondas elásticas num campo eletromagnético (geomagnético), onde essa propagação gera pequenas alterações geomagnéticas que se propagam, não com a velocidade da luz, mas ``acompanhando'' a onda elástica mantendo a velocidade desta última. 

A teoria é essencialmente uma combinação de elasticidade infinitesimal e teoria eletromagnética linearizada, e para torna o texto o mais auto-didata possível, serão apresentados nos capítulos \ref{sec.fund_eletr} e \ref{sec.fund_elast} os principais conceitos e definições acerca das teorias básicas sobre eletromagnetismo e elasticidade. 

Esta monografia faz parte de um conjunto de pesquisas que objetivam desenvolver um novo modelo matemático-computacional para descrever os fenômenos que envolvem a propagação simultânea de ondas eletromagnéticas e elásticas em subsuperfície, de modo que tal levantamento possa ser usado para aprimorar as técnicas de exploração de petróleo ou outro bem mineiral.