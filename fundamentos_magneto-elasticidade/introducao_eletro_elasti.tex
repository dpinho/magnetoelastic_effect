\chapter{Introdu\c{c}\~ao e Objetivos}
Esta monografia tem como finalidade principal o detalhamento da teoria que envolve o acoplamento de ondas eletromagn\'eticas e ondas el\'asticas que se propagam em meios estratificados, e homog\^eneo por camada, no subsolo terrestre. O desenvolvimento dessa teoria segue o modelo apresentado por \cite{erigen_1963} que trata da propaga\c{c}\~ao de ondas el\'asticas num campo eletromagn\'etico (geomagn\'etico), onde essa propaga\c{c}\~ao gera pequenas altera\c{c}\~oes geomagn\'eticas que se propagam, n\~ao com a velocidade da luz, mas ``acompanhando'' a onda el\'astica mantendo a velocidade desta \'ultima. 

A teoria é essencialmente uma combinação de elasticidade infinitesimal e teoria eletromagnética linearizada, e para tornar o texto o mais auto-suficiente poss\'ivel, serão apresentados nos capítulos \ref{sec.fund_eletr} e \ref{sec.fund_elast} os principais conceitos e definições acerca das teorias básicas sobre eletromagnetismo e elasticidade importantes para o efeito magneto-el\'astico.

Esta monografia faz parte de um conjunto de pesquisas que objetivam desenvolver um novo modelo matemático-computacional para descrever os fenômenos que envolvem a propagação simultânea de ondas eletromagnéticas e elásticas em subsuperfície, de modo que tal levantamento possa ser usado para aprimorar as técnicas de exploração de petróleo ou outro bem mineral.