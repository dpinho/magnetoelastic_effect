\chapter{Conclusões e Trabalhos Futuros}

Muitas pesquisas v\^em sendo realizadas no sentido de efetuar simula\c{c}\~oes num\'erico-computacionais que possam descrever diversos fen\^omenos f\'isicos relacionados \`a prospec\c{c}\~ao de petr\'oleo ou outro bem mineral, assim como fen\^omenos f\'isicos relacionados a terremotos ou que se aplicam a outros objetos de estudo. Essas simula\c{c}\~oes s\~ao ainda confrontadas com experimentos de campo na busca por consist\^encia entre essas duas faces do desenvolvimento de uma teoria. Numa oportunidade futura vamos desenvolver de forma anal\'itico-matem\'atica as EDP's da magneto-elasticidade,  e em seguida criar um algoritmo computacional capaz de efetuar simula\c{c}\~oes que nos auxiliem a estudar o efeito magneto-el\'astico, bem como entender e expandir a teoria geral que trata da intera\c{c}\~ao entre mec\^anica do cont\'inuo e eletromagnetismo em explora\c{c}\~ao de petr\'oleo.

O efeito magneto-el\'astico \'e descrito matematicamente pelo conjunto de EDP's formado pelo sistema de Maxwell e pelo sistema de Lam\`e, os quais s\~ao utilizados no estudo da propaga\c{c}\~ao acoplada de ondas s\'ismicas e eletromagn\'eticas na subsuperf\'icie terrestre. O acoplamento foi caracterizado, primeiramente, pela varia\c{c}\~ao que a for\c{c}a de Lorentz provoca no deslocamento do meio condutivo, simbolizado pela adi\c{c}\~ao da parcela referente \`a esta for\c{c}a na equa\c{c}\~ao do movimento de Cauchy. Segundo, pela a altera\c{c}\~ao eletromagn\'etica gerada pela passagem de uma onda s\'ismica que faz um meio condutivo oscilar no campo geomagn\'etico, simbolizada pela adi\c{c}\~ao desta varia\c{c}\~ao \`a lei de Amper\`e-Maxwell. Estamos estudando o caso parcialmente acoplado no espaco 3D, mas desejamos analisar tamb\'em o acoplamento total considerando tanto o espa\c{c}o 3D como o espa\c{c}o 1D, esperando que os desenvolvimentos e resultados em cada caso possam se complementar mutuamente.

Algumas hip\'oteses de ordem f\'isica, como o regime quasi-estacion\'ario por exemplo, foram necess\'arias para simplificar o modelo, linearizando as equa\c{c}\~oes e possibilitando o desenvolvimento anal\'itico das mesmas. Sendo assim, buscaremos pelas solu\c{c}\~oes dessas EDP's raciocinando basicamente com duas alternativas. Uma delas \'e trasnsformar as EDP's em EDO's utilizando ferramentas como as transformadas laterais de Fourier, transformadas de Hankel e mudan\c{c}as de eixos coordenados, escrevendo as equa\c{c}\~oes num formato onde \'e poss\'ivel aplicar um metodo matricial espec\'ifico para estudo de propaga\c{c}\~ao de ondas em meios estratigr\'aficos. Outra alternativa \'e reescrever as equa\c{c}\~oes em coordenadas cil\'indricas e fazer uso da hip\'otese de isotropia do meio de propaga\c{c}\~ao para reduzir as dimens\~oes do problema e obter as EDO's nas quais o m\'etodo matricial \'e aplicado.